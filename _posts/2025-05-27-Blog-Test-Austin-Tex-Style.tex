\documentclass[12pt]{amsart}
\usepackage{austin}


\begin{document}

\section{Test of Austin's Custom LaTeX Style}

This document tests various custom commands and environments from austin.sty.

\subsection{Custom Math Commands}

Here are some custom math symbols:
\begin{itemize}
    \item Complex numbers: $\C$
    \item Natural numbers: $\N$
    \item Rational numbers: $\Q$
    \item Real numbers: $\R$
    \item Integers: $\Z$
\end{itemize}

Some custom operators: $\cis(\theta) = \cos(\theta) + i\sin(\theta)$

The least common multiple: $\lcm(12, 18) = 36$

Automorphism group: $\Aut(G)$

\subsection{Custom Math Shortcuts}

Cube root: $\cbrt{27} = 3$

Floor and ceiling: $\floor{3.7} = 3$ and $\ceil{3.2} = 4$

Boxed result: $\bb{E = mc^2}$

Fraction shortcut: $\fr{a}{b} = \frac{a}{b}$

Vector notation: $\bv{v} = \V{x \\ y \\ z}$

\subsection{Theorem-Like Environments}

\begin{theorem}[Fundamental Theorem of Calculus]
If $f$ is continuous on $[a,b]$ and $F$ is an antiderivative of $f$, then
$$\intab{f(x) \, dx} = F(b) - F(a)$$
\end{theorem}

\begin{lemma}[Zorn's Lemma]
Every partially ordered set in which every chain has an upper bound contains at least one maximal element.
\end{lemma}

\begin{corollary}
Every vector space has a basis.
\end{corollary}

\begin{proposition}[Cauchy-Schwarz Inequality]
For any vectors $\bv{u}, \bv{v} \in \R^n$:
$$|\langle \bv{u}, \bv{v} \rangle| \leq \|\bv{u}\| \|\bv{v}\|$$
\end{proposition}

\begin{definition}[Group]
A group is a set $G$ together with a binary operation $\cdot: G \times G \to G$ satisfying:
\begin{enumerate}
    \item Associativity: $(a \cd b) \cd c = a \cd (b \cd c)$
    \item Identity: $\exists e \in G \st a \cd e = e \cd a = a$ for all $a \in G$
    \item Inverses: For each $a \in G$, $\exists a^{-1} \in G \st a \cd a^{-1} = a^{-1} \cd a = e$
\end{enumerate}
\end{definition}

\begin{example}[Symmetric Group]
The symmetric group $S_n$ consists of all permutations of $n$ elements. It has order $n!$.
\end{example}

\begin{remark}
The notation $\Hom(G, H)$ denotes the set of all homomorphisms from $G$ to $H$.
\end{remark}

% \begin{exercise}
% Prove that every finite cyclic group of order $n$ is isomorphic to $\Zp$.
% \end{exercise}

\subsection{Probability and Statistics}

Expected value: $\Ex{X} = \sum_{x} x \cdot P(X = x)$

Variance: $\Var{X} = \Ex{X^2} - \Ex{X}^2$

Covariance: $\Cov{X}{Y} = \Ex{XY} - \Ex{X}\Ex{Y}$

Bernoulli distribution: $X \sim \Bern{p}$

Binomial distribution: $Y \sim \Binom{n}{p}$

\subsection{Code Listings}

\begin{lstlisting}[language=Python]
def fibonacci(n):
    if n <= 1:
        return n
    return fibonacci(n-1) + fibonacci(n-2)
\end{lstlisting}

\subsection{Problem Environment}

\begin{problem}
Prove that $\sqrt{2}$ is irrational.
\end{problem}

\begin{subproblem}
Assume $\sqrt{2} = \fr{p}{q}$ where $p, q \in \Z$ and $\gcd(p, q) = 1$.
\end{subproblem}

\begin{subproblem}
Derive a contradiction.
\end{subproblem}

\end{document} 
