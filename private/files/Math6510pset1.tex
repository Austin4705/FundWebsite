\documentclass[12pt]{amsart}
\usepackage{austin}

\title{Math 6510}

\begin{document}
  \maketitle
  \begin{problem}
    \begin{claim}[$\gamma$ is surjective]
      Now $\forall c'\in C'$, $\delta$ being an isomorphism implies there exists $d'\in D\st \delta(d)=d'$. 
      Now exactness implies that $g_4(d')=g_4\cdot g_3(c')=0 \implies$ that $\delta\cdot f_4(b)=0 \implies$ since $\epsilon$ is an isomorphism that $f_4(d)=0\implies$ by exactness $d\in img(f_3)$. 
      Let $d=f_3(c)\implies g_3(c')=d' = \delta \cdot f_3(c) = g_3 \cdot \gamma(c)$. 
      Hence $c'-\gamma(c)\in ker(g_3) \implies$ by exactness that $d=f_3(c) \implies$ that $c'-\gamma(c)\in ker(f_3)$ so by exactness 
      $\exists b'\st c'-\gamma(c)=g_2\cdot \beta(b) \implies$ by commutivity that $c'-\gamma(c)=\gamma\cdot f_2(b) \implies c'=\gamma(c+f_2(b))$ Hence surjectivity. 
    \end{claim} 

    \begin{claim}[$\gamma$ is injective]
      Now $\forall c\in C \gamma(c)=0$ implies that $g_3\cdot \gamma(c)=0$ which implies by communativity that $\delta\cdot f_3(c)=0 \implies$ by $\delta$ bijective that $f_3(c)=0$\dots

      By exactness $\exists b\in B\st c=f_2(b)\implies \beta(b')=\gamma\cdot f_2(b)=\gamma(c)=0$ which implies by exactness that $\exists a'\in A'\st b'=f_1(a')$. $\alpha$ bijective implies that $\exists a\in A\st a'=\alpha(a)\implies \beta(b)=f_1\cdot \alpha(a)=\beta\cdot f_1(a)$ so $b-f_1(a)\in ket(\beta)$ which implies since $\beta$ bijective that $b=f_1(a)\implies$ by exactness that $c=f_2(b)=0$ hence injectivity. 
    \end{claim}
    Hence a structure preserving bijection so isomorphism \qed
  \end{problem}

  \newpage
  \begin{problem}
  \begin{enumerate}
    \item Recall that an $n$ sphere is defined as 
      \[\Sph^n \coloneq \{x\in\R^n\,|\,\ab{x}=1\}\]
      \begin{claim}[$\Sph^n$ is homeomorphic to the $n+1$-simplex]
        Let $V=\{0,1,2,\cdots,n+1\}$. Then represent each vertex as the standard basis vectors $e_0,\cdots, e_{n+1} \in \R^{n+2}$ so 
        \[\Delta^{n+1} \coloneq \{(x_0\cdots x_{n+1})\in \R^{n+2}| x_i \geq 0 | \sum x_i = 1\}\]. 

        Then the $\partial \Delta^{n+1}$ consists of all faces of $\Delta^{n+1}$. Now since $\Delta^{n+1}$ is a (n+1) manifold with boundary implies that $\Delta^{n+1}$ is homeomorphic to $D^{n+1}$. This implies the boundary is homeomorphic to $\Sph^n$ and so the boundary of the $\Delta^{n+1}$ simplex is homeomorphic to $\Sph^n$. 
      \end{claim}

      This intuitavely makes sense as $\Sph^1$ requires 3 simplicies to describe and $\Sph^2$ requires 4 to describe.  

    \item For any $n$ points there exists a $n-1$ plane that they all lie on. Therefore we need $n+1$ points at the least to define a simplicial complex that has volume in $n$ dimensions, which is the number of points we used in the previous question.    

    \item For $\Sph^0 = \{-1,1\}$ we get that it has exactly 2 connected compotnents so $H_0\Sph^0 = \Z^2, H_n\Sph^0 = 0, 1\leq n$. 

      For $\Sph^n, 1\leq n$, we get that $H_0(\Sph^n=\partial \Delta^{n+1}) = \frac{ker(d_0)}{im(d_1)} = \frac{C_0}{im(d_1)}$ so since the dinnferes for $im(d_1)$ imply there is one conected component, it is equal to $\Z$. 

      Similarly, for $H_n(\Sph^n)$ since there are no $n+1$ faces in $\Sph^n$, we get that $H_n(\Sph^n)$ is nonzero and therefore $\Z$. 

      In $1 \leq k < n$, $H_n(\Sph^n)$, there are no holes in $S^n$ outisde of $n$ so $H_k(\Sph^n)=0$.
    

    The pattern then matches to $H_k(\Sph^n) = \begin{cases}
      \Z & \text{ $k=0$ or $k=n$ } \\ 
      0 & \text{ otherwise}
    \end{cases}$. Thus for $\Sph^1$ it is $\Z, \Z, 0,\cdots$, $\Sph^2$ it is $\Z,0,\Z,0,\cdots$ and for $\Sph^3$ it is $\Z, 0,0,\Z,0,\cdots$. \qed 
  \end{enumerate}
  \end{problem}
  
  \newpage
  \begin{problem}
  \begin{enumerate}
    \item Torus: The construction of a torus can be created by creating a triangular prisim and then embedding a triangle inside that connects each edge of the traingle to the same edge on the top and bottom of the triangular prisim.
    
    I think you can use less points for a torus by having two triangles one embedded in another and an extrenous point, although I am not sure how to construct it. 

    \item $\R P^2$: Create the simplicial complex for $\Sph^2$ (a tetrahedron). Then embedd a vertex in each face and connect each that vertex to the opposite facing point in the tetrahedron. This is because $\R P^2$ is homeomorphic to $\Sph^2$ mod the equivlanece relation that identifies opposite points as the same. I think this is the mininum mumber of points needed. 

    \item Klein Bottle: A pyramid  with the bottom face removed and a series of traigular faces that wrap around onto the back face. 

      I am not sure about the mininum number of points needed. 
  \end{enumerate} 
  \end{problem}

  \newpage
  \begin{problem}
  \begin{enumerate}
  \item  
  \begin{claim}[Existence of an identity morphism]
    For any object $m$, There exists the identity morphism defined as $id_A \coloneq I_m$ where $I_m$ is the identity matrix of size $m$ such that given $f: m\to n$ represented as a $n\times m$ matrix then $f \circ id_m = fI_m = f$ and given $f: l\to m$ represetned as a $m\times l$ matrix, $id_m \circ f = I_m f = f$ Thus there exists an indentity morphism. \qed
  \end{claim}
  \begin{claim}[Composition is associative]
    Given three morphisms between nonnegative objects $n, l, k, m$ we define 
    \[h:n\to l,\, g:l\to k,\, f:k\to m\]
    Which implies that $h$ is a $l\times n$ matrix, $g$ is a $k \times l$ matrix, and $f$ is a $m\times k$ matrix. Thus 
    \[f\circ (g\circ h) = f(gh)\]
    \[(f\circ g)\circ h = (fg)h\] 
    And since matrix multiplication on the rationals is trivially known to be associative then the composition of morphisms is also associatve.\qed
  \end{claim}
  Thus since we have a collection of objects, a set of morphisms between obejcts (desribed in question), shown existence of an indentity morphism, and that composition of morphisms is associative, this obejct is indeed a category. \qed 
    \item 
      Define the functor $F(m)\coloneq \Q^m, G(V)\coloneq dim(V)$. 

      %Then $A\in C\implies m\mapsto m$ for $id_A \implies F(id_A)=\Q^m$. Then $F(A)$ maps $m\to \Q^m$ and $id_B, B\in Vect_\Q$ maps $\Q^m\to Q^m$ so $F(id_A)=\Q^m=id_{F(A)}$. 
      %
      %Then $B\in Vect_\Q$ given $id_B$ maps $\Q^m\to \Q^m$ which implies that $G(id_B)=G(\Q^m)=m$. $id_A,\,A\in C \implies id_A$ maps $m\to m$ which implies $id_{G(B)}=id_{A}=m$ where $A = G(B)$. Hence the identity axiom is defined for functors $F, G$. 
      %
      Now given $A: m\to n$ and $B: l\to m\implies F(A\circ B) = F(n)= \Q^n$ and $F(A)\circ F(B)$ maps $\Q^l\to \Q^n$ which is equal to $F(A\circ B)$. 

      Now given $G$ and morphism $f: V\to W$ use axiom of choice to construct an arbitrary basis for $V$. Then there exists some unique linear map (matrix) that maps $V\to W$ to the respective chosen basis. This implies that $G(f): G(V)\to G(W)$ is a morphism in $C$ which implies since composition of matrix multiplication responds to the underlying linear maps that $G$ is a functor. \qed 

  \end{enumerate}
  \end{problem}

  \newpage
  \begin{problem}
  \begin{enumerate}
    \item 
    Suppose that $x\in NP$. Then $x$ is finite with some cardinality $n$ by definition and consits of the elemnts $\{p_0,\cdots,p_n\}$ where there is a ordering $p_1 < p_2 < \cdot < p_n$. Now removing $p_1$ or $p_n$ gives the set $\{p_2,\cdot, p_n\}$ or $\{p_1,\cdot, p_{n-1}\}$ which trivially has the ordering $p_2 < p_3 < \cdots < p_n$ or $p_1 < p_2 < \cdots < p_{n1}$. Thus they must also be in $NP$.
    Now removing and $p_i,\, 1 < i < n$ implies that there is a set $\{p_1,p_2,\cdots p_{i-1},p_{i+1},\cdots p_n\}$. Now since it is known that $p_{i-1} < p_i$ and $p_i < p_{i+1}$ then by transitivity $p_{i-1} < p_{i+1}$ which implies there is a ordering $p_1 < \cdots p_{i-1} < p_{i+1} < p_n$. 
    Thus there is a ordering and this set must be in $NP$. Removing 1 element from a set with cardinality 2 is also solved since subsets with cardinality 1 are trivially ordered. Thus every subset of $x$ containing exactly 1 less element than $x$ is also in $NP$. 
    Continuting this process recursively results in finite temrination since $x$ is finite and also results in every subset of $x$ being generated by recurisvely removing a single element in $x$ that is not in the subset implying this element $x'$ is also in $NP$ and recursively repeating this process on $x'$ with an element in $x'$ but not in the desired subset of $x$.
    Thus every subset of $x\in NP$ is also in $NP\forall x \implies NP$ is a simplicial complex.\qed 
  
  \item Given a morphism $f:P\to Q$ in $PoSet$, define the functor $N$ as $N(f): N(P)\mapsto N(Q)$. 

    Now since $id_p: P\to P$ is the identity then $N(id_P)$ is the identitiy on $P$. Thus $N(id_p)=id_{N(p)}$. 

    Now since $N(g)\circ N(f)(p) = N(g)(N(f)(p)) = N(g)(f(p)) = g(f(p))$ Hence assosiativity is preserved and it is a functor. 

    Then $N(f)$ is a simpliial map since $N(g\circ f)(P)$ returns $N(g\circ f)(P) = N(g)\circ N(f)(P)$ which is a set consisiting of the finite sets of $Q$ with ordering $p_0 \leq \cdots \leq p_n$ which implies from the first part of the question it is a.  

    \item Since $x\leq y$ becomes $y\leq x$ in $P^{op}\implies P^{op}(A,B) = P(B,A)$. This means there is a isomorphism between $P^{op}(A,B)$ and $P(B,A)$ which implies that the cardinality of $|NP|$ and $|NP^{op}|$ are equivalent. \qed  

    \item Define $cf\coloneq f\coloneq\{f(p_0)\cdots f(p_k)\}$. Then 
     
    $cf(x)=f(x)=\{f(p)| p\in x\} \subseteq f(x')-cf(x')$. Thus it is order preserving. 

    This implies that $c(id_L)(x) = \{id_L(v)| v\in x\} = x$. So idenitiy map is preserved. and
    \[(cg)(cf(x)) = (cg)(\{f(p) | p\in x\}) = \{g(f(p)) | p\in x\} = (g\circ f)(x)= c(g\circ f)(x)\]

    Thus associativity is defined so it is a functor. Since it is order preserving it follows reflexivity and transitive properties so it is a poset. 

  \item Now since the property that all subsets of all elements are in a simplex, mapping to the max vertex in the order implies that all subset verticies are also included in the map so it is simplicial.
  \end{enumerate} 
  \end{problem}
\end{document}
