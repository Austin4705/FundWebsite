\documentclass[aps,prb,twocolumn]{revtex4-2}
\usepackage{austin}
\usepackage[english]{babel}


\renewcommand{\qedsymbol}{$\blacksquare$}
\begin{document}
  \title{Phys 7687: Quantum Anomalous Hall Effect}
  \author{Austin Wu}
  \maketitle
  \tableofcontents

  \section{Introduction}
  The Quantum Anomalous Hall Fffect or QAHE is an effect in which certain topological insulator materials that conduct electricity on their surface but not in their bulk have broken time-reversal symmetry which then causes quantized values of hall conductance, leading to discrete conductance levels of integer multiples of $\frac{e^2}h$. In this expository paper, we will delve into the principles underlying the QAHE, explore the materials and experimental results that exhibit this effect, do first principles calculations, and discuss the potential applications and challenges in harnessing this quantum phenomenon for technological advancements.



  \section{Theoretical Derivation of Quantum Anomalous Hall Effect}
    \subsection{Basic Mathematical Background for Chern Numbers and Chern--Simons Theory}
      To begin, lets formally define some of the topological obejcts that we will be working with.
      \begin{definition}[Smooth manifolds] 
        Recall that a subset $\mathcal{M}\subset \R^n$ is a smooth ($C^\infty$) $k$ dimensional manifold if locally it is the graph of a $C^1$ mapping $f$ expressing $n-k$ variables as functions of the other $k$ variables. \cite{hubbard_vector_2009}
      \end{definition}
      Given smooth manifolds $X,Y\subset\mathbb{R}^n$, define an equivalence
      relation on $\mathrm{C}^\infty(X,Y)$ by declaring a homotopy class.
      \begin{definition}[Homotopy equivlanece]
        Let $X,Y$ be topological spaces.  Two continuous maps $f,g\colon X\!\to\! Y$ are \emph{homotopic}, written $f\simeq g$, if there exists a continuous $H\colon X\times[0,1]\!\to\!Y$ with $H(x,0)=f(x)$ and $H(x,1)=g(x)$. The map $H$ is called a \emph{homotopy} between $f$ and~$g$. We say that the \emph{homotopy class} of $f$ is 
        \[ [f]\;=\;\{g\in C(X,Y)\mid g\simeq f\}. \]
      \end{definition}

      \begin{definition}[Homotopy-invariant quantity]
        A map $I\colon[X,Y]\to \mathcal{A}$ into some algebraic set $\mathcal{A}$ is called a \emph{topological invariant} of maps $X\to Y$ if $I([f])=I([g])$ whenever $f\simeq g$.  When $X$ itself varies one usually asks for invariance under homeomorphism of spaces.
      \end{definition}

      Now that we have these definitions out of the way, we can start defining topological invariants relevant to band thoery. We start by

      \begin{definition}[Chern Class]
        Let $E\!\xrightarrow{\;\pi\;}M$ be a rank-$r$ complex vector bundle over a closed orientable $2n$-manifold $M$. Pick any connection $\nabla$ with curvature two-form $F\in\Omega^2(M,\mathfrak{u}(r))$. The total \emph{Chern form}
        \[ c(E,\nabla)\;=\;\det\!\bigl(\mathbf 1+\tfrac{i}{2\pi}F\bigr) \;=\;1+c_1(F)+\dots+c_r(F) \]
        has closed components $c_k(F)\in\Omega^{2n}(M)$; their de Rham cohomology classes $c_k(E):=[c_k(F)]\in H^{2n}(M;\Z)$ are independent of~$\nabla$ and define the \emph{$n$-th Chern class}.
      \end{definition}

      Note however since it is defined for $2n$ manifiolds, we can only define chern classes for even dimension manifiolds. Thus note that for the $n=1,\dim(M)=2$ case we get the first chern class to be
      \[ c_1(F)\;=\;\frac{i}{2\pi}\operatorname{tr}F. \]
      Which implies that its integral
      \[ \mathrm{C}_1(E)\;=\;\int_M c_1(F)\;\in\;\Z \]
      is called the first Chern number of~$E$. Naturally, this leads us to also ask if there are similar invariants for $3$ manifiolds. 

      \subsubsection{Chern--Simons forms on odd--dimensional manifolds}

      For odd-dimensional $M$ the ordinary Chern numbers vanish,
      but one can construct secondary invariants whose exterior
      derivative reproduces a higher Chern form.  Given a connection
      $A$ with curvature $F=dA+A\wedge A$ on a rank-$r$ complex bundle,
      define the $(2m\!-\!1)$-form
      \begin{multline}
        \operatorname{CS}_{2m-1}(A) =\frac{m}{(2\pi)^m(m-1)!}\\
      \int_0^1\!\operatorname{tr} \!\Bigl(A\wedge  (t\,dA + t^{2} A\wedge A)^{m-1}\Bigr)dt
      \end{multline}
      Its exterior derivative satisfies
      \( d\,\operatorname{CS}_{2m-1}(A)=\tfrac1{(2\pi)^{m}}\operatorname{tr}(F^{m}), \)
      so on a closed $(2m\!-\!1)$-manifold $M$
      \(\displaystyle \Theta_{2m-1}[A]\;=\;\int_{M}\!\operatorname{CS}_{2m-1}(A) \)
      is a topological invariant.

      Thus setting $m=2$ gives
      \begin{equation}
        \operatorname{CS}_{3}(A) \;=\;\frac{1}{4\pi}\,\operatorname{tr} \bigl(A\wedge dA + \tfrac23 A\wedge A\wedge A\bigr). \label{eq:CS3}
      \end{equation}
      wth its derivative being 
      \(d\,\operatorname{CS}_{3}=\tfrac{1}{8\pi^{2}}\operatorname{tr}(F\wedge F)
       = c_{2}(F)\),
      or the second Chern form. This being said, integer Quantum Anomalouse Hall Effect can be described with just the first chern number in a 2D crystal.

  \section{Theory behind Quantum Anomalous Hall Effect}

      Now that we have established the topological invariants that become relevant, we can use them to characterise the Quantum Anomalous Hall Effect.       
      \subsubsection{Berry curvature as a momentum--space magnetic field}
      For a single Bloch band the semiclassical equations of motion read            
      \begin{equation}
        \dot{\mathbf r} = \nabla_{\mathbf k} \varepsilon(\mathbf k)+\mathbf E \times \bm{\Omega}(\mathbf k)
      \end{equation}
      \begin{equation}
        \dot{\mathbf k} = -e\mathbf E
      \end{equation}
      where $\bm{\Omega}=\nabla_{\mathbf k}\times \mathbf A$ is the Berry curvature. Th key reason we see this occruence is because the second term acts exactly like the Lorentz force in momentum space, with $\bm{\Omega}$ playing the role of an effective magnetic field.

      Now since $\bm\Omega$ is odd under time-reversal, a quantized hall effect appears only after $\mathcal{T}$ is broken. We can do this in a conventional hall effect by adding a external magnetic field, but in order to have an anomalous setup we must have some intrisitc mechanism (say ferromagentic order). Thus the challenge becomes creating a system in which we can break time symetry.

      \subsubsection{Quantised transverse conductivity}
      Integrating the anomalous velocity over all occupied states while including the Fermi--Dirac weight yields the Kubo--Berry formula
      \begin{equation}
      \sigma_{xy} = -\frac{e^{2}}{\hbar} \sum_{n\in\text{occ}} \int_{\text{BZ}} \frac{d^{2}k}{(2\pi)^{2}}, \Omega^{(n)}_{z}(\mathbf k).
      \end{equation}
      Because $\Omega^{(n)}_{z}=\partial_{k_{x}}A_{y}-\partial_{k_{y}}A_{x}$ is a
      total derivative, the integral can change only by integer multiples of
      $2\pi$; the result is the Chern number $C$. This creates $\sigma_{xy}=C e^{2}/h$. Importantly, this quantisation is independent of material details such as band dispersion or sample geometry as long as the bulk energy gap remains open. This is fundnamentally where we get our topological principles from.

      \subsection{Topological Phase Transitions and Criticality}
      We know that a Chern number can change only when the bulk gap closes as that would be the time its topological structure changes. This implies that the quantum phase transition between a trivial ($C=0$) and QAH ($C\!\neq\!0$) state is controlled by a gapless Dirac point. If we wer to write generic two-band Bloch Hamiltonian as
      \[ H(\mathbf k,m)=d_x(\mathbf k,m)\,\sigma_x +d_y(\mathbf k,m)\,\sigma_y +d_z(\mathbf k,m)\,\sigma_z, \]
      one finds a critical mass $m=m_c$ for which
      \[d_x=d_y=d_z=0\]
      at some $\mathbf k=\mathbf k_c$.  Expanding to linear order gives 
      \begin{equation}
        H_{\text{crit}}=v_x q_x\sigma_x+v_y q_y\sigma_y +m_r\sigma_z
        q\equiv k-k_c
      \end{equation}
      with $m_r\propto m-m_c$.The parity of the mass term $m_r$ fixes the jump 
      \[ \Delta C=\frac12\bigl[\operatorname{sgn}(m_r^{+})- \operatorname{sgn}(m_r^{-})\bigr]=\pm1. \]

    \subsection{Bulk–Edge Correspondence and Chiral Edge Modes}
      Bulk edge correspondence states that we can also find topological invariants in the bulk band sturcture that give edge states at the boundries. The Chern number not only fixes $\sigma_{xy}$ but also dictates the existence of $|C|$ chiral edge channels. Comparing the spectral flow of a Dirac Hamiltonian $H(\hat k)$ interpolated across a spatial domain wall where the Dirac mass $m(y)$ changes sign
      \begin{equation}
        H = v\,k_{x}\sigma_{x} + v\,k_{y}\sigma_{y} + m(y)\,\sigma_{z}, \;
        m(-\infty)=-m(\infty)\!.
      \end{equation}
      Solving the Dirac equation shows a single, unidirectional zero mode
      \[ \psi_{0}(x,y)\propto e^{-\!\tfrac1v\!\int_{0}^{y} m(y')\,dy'}\,e^{ik_{x}x}, \]
      whose dispersion is $E(k_{x})=v\,k_{x}$, confirming the edge hosts a
      dissipation-less 1D channel.      

  \section{Material Realizations}
    \subsection{Haldane Model on a Honeycomb Lattice}
      % The Haldane model on a honeycomb lattice was the first theoretical relization of the Qunatum Anomalous Hall Effect. It is defined by the honeycomb lattice
      % %TODO: input lattice vectors
      % Resulting in a signle particle hamiltonian represetend by
      % \[h_0(\hat k)=\V{0&g(\hat k)\\g^*(\hat k)&0}\]
      % when the nearest neighbor hopping strength $t$ is set to 1. This makes the dispersion of 
      % Since he total flux through the primitive cell is zero, so time reversal symmetry is broken locally but preserved on average. Thus we can derive its toplogical properties and show that CHERN TODO leads to the Anomalous Quantum Hall Effect. 
      % In order to show its topologhical properties, we must use the berry connection
      % \[h_o(\hat k)=\V{0&g(\hat k)\\g^*(\hat k)&0}\]
      % where we get
      % Now we know from the tight binding model of graphene that at the $K,K'$ points we get areas between the band with linear disperstion.
      % Note now since the Haldane Model is mostly created by the 2D honeycomb lattice, we can create a real world example of this by using Graphene Paired with \cite{LectureNotesChapter3}

      The Haldane model, proposed by F.D.M. Haldane in 1988, was the first theoretical realization of the Quantum Anomalous Hall Effect \cite{LectureNotesChapter3}. It demonstrates that a quantized Hall effect can occur in a periodic system even in the absence of a net external magnetic field. The model is defined on a honeycomb lattice.
      
      Considering only nearest-neighbor (NN) hopping with strength $t_1$ (set to $t_1=1$ for normalization as in ), the single-particle tight-binding Hamiltonian in momentum space is represented by a $2 \times 2$ matrix, where the basis corresponds to the two sublattices A and B
      \[ h_0(\hat{k}) = \begin{pmatrix} 0 & g(\hat{k}) \\ g^*(\hat{k}) & 0 \end{pmatrix} \]
      where $g(\hat{k})$ is given by
      \[ g(\hat{k}) = t_1 \sum_{j=1}^{3} \exp(i \hat{k} \cdot \hat{a}_j) \]
      This Hamiltonian $h_0(\hat{k})$ is essentially the Hamiltonian for graphene, which features two inequivalent Dirac cones (points where the valence and conduction bands touch linearly) at the $K$ and $K'$ points of the Brillouin zone. At these points, the dispersion relation is approximately linear, resembling that of massless relativistic Dirac fermions.
      
      To realize the QAHE, Haldane introduced two modifications to this basic graphene model to open a gap at the Dirac points and break time-reversal symmetry, without requiring a net magnetic flux through the unit cell. The first of which was a sublattice energy offset which adds $M\sigma_z$ to the Hamiltonian, where $\sigma_z$ is the Pauli z-matrix. This term alone opens a gap but results in a trivial insulator. The other was Next-Nearest-Neighbor (NNN) Hopping ($t_2$). This term breaks time-reversal symmetry locally. The NNN hopping vectors are defined as:
          \begin{align*}
              \vec{b}_1 &= (0, \sqrt{3}) \\ % a2 - a3
              \vec{b}_2 &= \left(-\frac{3}{2}, -\frac{\sqrt{3}}{2}\right) \\ % a3 - a1
              \vec{b}_3 &= \left(\frac{3}{2}, -\frac{\sqrt{3}}{2}\right) % a1 - a2
          \end{align*}
      The full Haldane model Hamiltonian is then:
      \begin{multline}
        h(\vec{k}) = \left( M + 2t_2 \sum_{j=1}^{3} \sin(\hat{k} \cdot \vec{b}_j) \right) \sigma_z \\
        + \text{Re}(g(\hat{k}))\sigma_x - \text{Im}(g(\hat{k}))\sigma_y
      \end{multline}
      
      The NNN hopping term breaks time-reversal symmetry ($h(\hat{k}) \neq \sigma_y h^*(-\hat{k}) \sigma_y$) and, in conjunction with the mass term $M$, can lead to a non-trivial band topology. 
      
      The dispersion relations for the two bands are 
      \[\mathcal{E}_{\pm}(\hat{k}) = \pm \sqrt{|g(\hat{k})|^2 + (M + 2t_2 \sum_j \sin(\hat{k} \cdot \hat{b}_j))^2}\]
      The Chern number can take integer values. For the Haldane model, it is found that $C$ can be $0, +1,$ or $-1$, depending on the parameters $M$ and $t_2$. If $C = \pm 1$, the system is in a topologically non-trivial phase and exhibits the QAHE. This means it has a quantized Hall conductance $\sigma_{xy} = C \frac{e^2}{h}$ without an external magnetic field. If $C = 0$, the system is a topologically trivial insulator.
      The Chern number remains quantized as long as the energy gap is open and can only change when the gap closes, signaling a topological phase transition. This typically occurs when $M = \pm 3\sqrt{3}t_2$ (under specific phase conditions for NNN hopping) at the K or K' points.
      
      Note now since the Haldane Model is mostly created by the 2D honeycomb lattice, we can create a real-world example of this by using Graphene paired with other materials or techniques to engineer the required effective magnetic fluxes and sublattice symmetry breaking 

    \subsection{Qi–Wu–Zhang (QWZ) square-lattice model}
      The QWZ model is generated by a tight binding model for a Chern insulator on a 2D square lattice given by 
      \[H(k)=\sin k_x \sigma_x + \sin k_y \sigma_y + (m+\cos k_x+\cos k_y)\sigma_z\]
      where $\sigma_i$ is a pauli matrix on spin/oribial psuedopsin and $m$ is the effective mass. Doing out the derivation we find that when the effetive mass is between $\ab|m|\le2$ we get a chern number of $\pm1$ which creates a topological insulator and quantized hall conductivity. If $m$ is outside this range we get that its chern number is $0$ which makes it a trivial insualtor.
      
    \subsection{Magnetically doped 3D-TI thin film}
      Originally proposed by Shou-Cheng Zhang\citeauthor{yu_quantized_2010}, the idea revlovles areound taking a 3D topological insulator we can then magentically dope it with magnetic elements in order to achieve the same effect with breaking time reversal symmetry. 

      As the sample gets asymtotically thin we can see a hybridzation of the top and bottom surface states, which creates a hamiltonian reprsetned by
      \[H(k)=v_F(k_x\sigma_y-k_y\sigma_x)+m(k)\sigma_z\]
      Similarly here the effective mass term can determine the Chern number and with a nonzero total Chern number we get a quantized hall conductance. 

  \section{Historical Context}
    \subsection{Classiacal Hall Effect}
      The original hall effect was disovered by Edwin Hall in 1879 uisng the derivation of the lorenz force, 
      \[F=q(E+v\times B)\]
      Thus in steady state, $F=0$ so 
      \[E_y = -v_xB_z\]
      Thus we get that the Hall voltage becomes 
      \[V_h = \frac{I_x B_z}{nle}\]
      with resistance
      \[R_H=\frac{E_y}{j_xB_z}\]
      although clearly this system requires a magnetic field and is not quatnized.
    \subsection{Quantam Hall Effect}
    In 1980, Klaus von Klitzing made the unexpected discovery that the Hall resistance exhibited exact quantization under certain conditions, known as Quantum Hall effect. There were plateaus in the Hall resistance (the ratio of Hall voltage to current) quantized in exact multiples of $\frac{\hbar}{e²}$. 

    The quantum Hall effect occurs because in a strong magnetic field at low temperatures, electrons in a two-dimensional system form Landau levels, or discrete energy states similar to the energy levels in atoms. As the gate voltage changes, the number of filled Landau levels changes, leading to the step-like behavior in conductance.
    \begin{equation}
      R_{xy} \;=\;\frac{V_{y}}{I_{x}}
      \;=\;\frac{h}{e^{2}\,\nu},
      \qquad
      \nu\in\mathbb{Z},
      \label{eq:intQHE}
    \end{equation}
    Here $\nu$ is the the number of fully occupied Landau levels.  In a perpendicular field $B_{z}$ the kinetic energy of free electrons is quantised into equally spaced Landau levels
    \( E_{n}=\hbar\omega_{c}(n+\tfrac12), \; \omega_{c}=eB_{z}/m. \)
    Note how each level carries a chern number associated with it. $C=1$, so $\nu$ filled levels give the TKNN integer $C_{\text{tot}}=\nu$ and the quantised Hall conductance $\sigma_{xy}=\nu\,e^{2}/h$.

    \subsection{To the anomalous hall effect}
      In a conventional quantum Hall system, electrons execute cyclotron orbits with their Landeau Levels. In the QAHE the momentum--space curvature replaces the real magnetic field. This means that an electron adiabatically circling the Brillouin zone picks up a Berry phase $2\pi C$. Thus the edge therefore supports the same chiral motion without an applied $B$-field and allows us to create a similar quantized theory without a magentic field if we can use momentum-space curvature.


  \section{Worked Example}
    In this section we derive the quantum anomalous hall effect for a 2D material. At zero temperature the hall conducitivty of a band insulator is given by 
    \[\sigma_{xy}=-\frac{e^2}h\sum_{n\in\mathrm{occ}}\int_{\mathrm{BZ}}\frac{d^2k}{(2\pi)^2}\Omega_2^{(n)}(k)\]
    Where $\Omega_z^{(n)}(k)=\partial_{k_x}A_y^{(n)}-\partial_{k_y}A_x^{(n)}$ and $A_i^{(n)}=i\ab<u_{nk}|\partial_{k_i}|u_{nk}>$, or the berry conenction.
    Now breakting $\mathcal{T}$ reversal makes 
    \[\mathcal{T}:\Omega_z^{(n)}(k)\mapsto-\Omega_z^{(n)}(-k)\]
    Notably if $\mathcal{T}$ were unbroken, $\sigma_{xy}$ would become 0. This implies a 2D mateiral needs a $\mathcal{T}$ breaking field such as magnetisim for an anomalous hall response.
    
    Now recall the general formula for a 2 band hamiltonian
    \begin{equation}
      H(k)=d(k)\cdot\V{\sigma_x\\\sigma_y\\\sigma_z}
    \end{equation}
    and energies $E_\pm=\pm\ab|d|$. Then let $\hat{d}$ be the unit vetctor of $d$. This then implies that 
    \[\Omega_z(k)=\frac12 \hat{d}\cdot (\partial_{k_x}\hat{d}\times\partial_{k_y}\hat{d})\]

    Now we let
    \[H(k)=v(k_x\sigma_y-k_y\sigma_x)+m\sigma_z\]
    where $m$ is the spin orbit induced mass. This implies that
    \[\Omega_z(k)=-\frac{mv^2}{2(v^2k^2+m^2)^{\frac32}}\]
    Which means that if we integrate over the $S_\times S_1$ manifold generated by the Brillouin zone (the torriodal nathre is generated by the inherent periodicity of the B.Z.), by the properties of chern numbers we get that
    \[\int \frac{d^2k}{2\pi}\Omega_z(k)=-\frac12\sgn(m)\equiv C\]
    Which directy implies that the quantized hall conducitvity is 
    \[\sigma_{xy}=C\frac{e^2}{h}\]

    Note that given a second dirac cone of the same sign mass, the chern number would double, giving a non quantized anomalous hall effect.

  \section{Experimental Results}
    \subsection{First Observation in Cr‐doped (Bi,Sb)\texorpdfstring{\textsubscript{2}}{₂}Te\texorpdfstring{\textsubscript{3}}{₃}}
      Chang \emph{et al.}\ \cite{chang_experimental_2013} Created the first realization of the anomalous hall effect by growing a 5-quintuple-layer film of Cr\textsuperscript{\,+3}‐substituted (Bi,Sb)\textsubscript{2}Te\textsubscript{3} by MBE on SrTiO\textsubscript{3}(111). Afterwards, they swept the Fermi level into the magnetically opened surface gap by gate tuning, yielding a quantised Hall plateau $R_{yx}=h/e^{2}$ with a drop of $R_{xx}$ below $50\;\Omega$ at $T \simeq 30\,$mK, fundnamentally showing qunatized hall conductance and a Chern-number $C=1$ insulator.  Hysteretic $R_{yx}(B)$ loops confirmed out-of-plane ferromagnetism and time-reversal symmetry breaking.

     This experiment confirmed the central theoretical prediction of Yu \emph{et al.}\ \cite{yu_quantized_2010} that a Chern‐insulating ground state can emerge when strong spin–orbit coupling is combined with spontaneous magnetisation.  The move from diluted to intrinsic ferromagnetism is rapidly elevating the operating temperature and metrological fidelity of QAH devices, laying the groundwork for dissipationless interconnects and compact resistance standards.

    \subsection{Thickness and Doping Dependence}
    The Tokura lab \cite{yasuda_quantized_2017} pushed the quantum–anomalous-Hall platform beyond fixed sample edges by writing magnetic domains inside a Cr-doped (Bi,Sb)\textsubscript{2}Te\textsubscript{3} Hall bar. A quintuple-layer film ($\sim\!10$ nm) shown to be able to be cooled to $T\simeq0.5$ K, where opposite out-of-plane magnetisations correspond to Chern numbers $C=\pm1$. This means that using the local field from a magnetic-force–microscope (MFM) tip, the team patterned micrometre-scale domains at will, thereby creating domain walls (DWs) with a Chern-number difference $\Delta C = 2$ between adjacent regions.
    
    % Thus Two-terminal measurements between voltage probes straddling an isolated DW yielded a conductance 
    % \[ G = \frac{e^{2}}{h}\quad(\text{within }<1\%),\]
    %
    % while the longitudinal resistance $R_{xx}$ across the same channel vanished below the noise floor. This is the hallmark of a single, dissipation-less chiral edge mode predicted by Landauer–Büttiker theory for a one-dimensional conduit with unit Chern number. Because the MFM tip can reversibly flip local magnetisation, the DW network acts as a rewritable “wiring diagram’’ inside the QAH insulator. This demonstrates programmable current dividers and interferometer-like geometries by sculpting multiple DW channels, showcasing a route to low-power, non-volatile topological electronics. The experiment proves that magnetic DWs can serve as lossless interconnects whose geometry is defined purely by the magnetisation texture, opening a new design space for chip-scale “topotronics’’ that leverages controllable Chern boundaries rather than lithographic edges.
    \subsection{Experimental Realization of Magnetic Doping}
      Deng \textit{et al.}\,\cite{deng_quantum_2020} achieved the first zero-field quantum-anomalous Hall (QAH) effect in a stoichiometric magnetic topological insulator—five-septuple-layer (5 SL) MnBi\textsubscript{2}Te\textsubscript{4} (MBT). This was a realization of magnetically doped 3D Ti thin films. Note that MBT is an A-type antiferromagnet, each septuple layer is ferromagnetic, but adjacent SLs couple antiparallel.  

      An odd septuple layer count therefore carries a net out-of-plane magnetisation that breaks time-reversal symmetry without introducing substitutional disorder. At $T = 1.4\,$K the Hall resistance quantised to $R_{yx}=h/e^{2}$ while the longitudinal resistance fell below $10\Omega$, indicating a Chern number $C=1$ insulator in the absence of any external field and creating a zero field quantization. 

      Similarly, a modest perpendicular field (${<}1\,$T) forced all SLs into a ferromagnetic alignment, making the QAH onset to $T_{\mathrm{QAH}}\approx 6.5\,$K—two orders of magnitude higher than in Cr-doped (Bi,Sb)\textsubscript{2}Te\textsubscript{3} and enducing robustness. 

  \section{Applications}
    \subsection{Low power electronics}
      The creation of materials that can exhibit the quantum anomlaous hall effect cal help us generate new low power electronis because the chiral edge states imply a system that conducts electricity with minimal dissipation without needing a magnetic field in order to exhibit this behavior.
    \subsection{Quantum Computing}
      QAHE potetnially has the ability to realize topological quantum computing due to chiral Majorana edge states.
  \bibliography{refrences}
  \bibliographystyle{apsrev4-2}
\end{document}
