\documentclass[12pt]{amsart}
\usepackage{austin}

\title{Pset 1}
\font\nullfont=cmr10

\author{Minh Pham}

\begin{document}
  \maketitle
  \begin{problem}
    \begin{subproblem}
      We know that for the transition function
      \begin{center}
      \begin{tabular}{|c|c|c|c|c|}
        \hline 
        s & a & A & B & C \\
        \hline
        A & N & $\frac12$ & $\frac12$ & 0 \\ 
        A & S & 0 & 1 & 0 \\ 
        B & N & $\frac12$ & $\frac12$ & 0 \\
        B & S & 0 & $\frac12$ & $\frac12$ \\ 
        C & N & 0 & 1 & 0 \\ 
        C & S & 0 & 0 & 1 \\
        \hline
      \end{tabular}
      \end{center}
      
      And for the reward function
      \begin{center}
      \begin{tabular}{|c|c|c|}
        \hline
        \textbf{State \( s \)} & \textbf{Action \( a \)} & \textbf{Reward \( r(s, a) \)} \\
        \hline
        A & N & 1 \\
        A & S & 1 \\
        B & N & -1 \\
        B & S & -1 \\
        C & N & 1 \\
        C & S & 1 \\
        \hline
    \end{tabular} 
    \end{center}
    \end{subproblem} 
    \begin{subproblem}
      Dynamic programming implies that 
      \[Q_h^*(s,a) = r(s,a) + \expect_{s'\sim P(\cdot|s,a)}\ab[V_{h+1}^*(s')],\, \pi_h^*(s) = \argmax_a Q_h^*(s,a).\]
      Thus since $Q^*_{H-1}(s,a)=r(s,a),\, \pi_{H-1}^*(s)=\argmax_a Q^*_{H-1}(s,a)$ and $V_{H-1}^*(s)=\max_a Q_{H-1}^*(s,a)$. 

      Now since $V_{H-1}^*(A,B,C) = \{1,1-1\}$, $\pi_{H-1}^*(A,B,C) = \{\frac NS,\frac NS, \frac NS\}$ thus using the chart above 
      \[Q_{H-2}^*(A,N) = 1 +\frac12-\frac12 = 1,\, Q_{H_2}^*(A,S) = 1-1=0\] so $\pi_{H-2}^*(A)=N$. 
      And 
      \[Q_{H-2}^*(B,N)= -1,\, Q_{H-2}^*(B,S) = -1\] so $\pi_{H-2}^*(B) = \frac NS$

      And \[Q_{H-2}^*(C,N) = 0,\, Q_{H-2}^*(C,S)= 2\] so $\pi_{H-2}^*(C)=S$ 

      Thus since $V_{H-2}^*(A,B,C)= \{1,-1,2\}$ implies that 
      \[Q_{H-3}^*(A,N) =-1,\, Q_{H_3}^*(A,S) = 1-1=0\] so $\pi_{H-3}^*(A)=N$. 
      And 
      \[Q_{H-3}^*(B,N)= -1,\, Q_{H-3}^*(B,S) = -\frac12\] so $\pi_{H-3}^*(B) = S$

      And \[Q_{H-3}^*(C,N) = 0,\, Q_{H-3}^*(C,S)= 3\] so $\pi_{H-3}^*(C)=S$ 
      This implies that $\pi^*_{H-3} =\{N,S,S\}$ is the optimal policy for our problem as for each timestep it is the optimal policy. Thus 
      \[\pi^*(A,B,C)=\{N,S,S\}\] 
    \end{subproblem}
    \begin{subproblem}
      We know that 
      \[V^\pi(A)=r(A,N)+\gamma\expect_{s'\sim P(\cdot |A,N)}\ab[V^\pi(s')] = 1+\gamma(\frac12V^\pi(A)+\frac12V^\pi(B))\]      
      and that 
      \[V^\pi(B)=r(B,S)+\gamma\expect_{s'\sim P(\cdot |B,S)}\ab[V^\pi(s')] =-1+\gamma(\frac12V^\pi(B)+\frac12V^\pi(C))\]
      and that 
      \[V^\pi(C)=r(C,S)+\gamma\expect_{s'\sim P(\cdot |C,S)}\ab[V^\pi(s')] =-1+\gamma V^\pi(C)\]
      So algebraicly solving for $V^\pi(C)$ we get
      \[V^\pi(C)==-1+\gamma V^\pi(C)\imples V_\pi(C)=\frac1{1-\gamma}\]
      And plugging this value into our expression for $V^\pi(B)$ implies that 
      \[V^\pi(B)=-1+\gamma(\frac12V^\pi(B)+\frac12\frac1{1-\gamma})\imples V^\pi(B) = \frac{3\gamma-2}{(1-\gamma)(2-\gamma)}\]
      And finally using this value into our expression for $V^\pi(A)$ implies that 
      \[V^\pi(A)= 1+\gamma(\frac12V^\pi(A)+\frac12\frac{3\gamma-2}{(1-\gamma)(2-\gamma)})\imples V^\pi(A) = \frac{\gamma(3\gamma-2) + 2(1-\gamma)(2-\gamma)}{(1-\gamma)(2-\gamma)^2}\]      

    \end{subproblem} 
    \begin{subproblem}
      Recall that 
      \begin{definition}[Bellman Optimality Equation]
        $Q(s,a) = r(s,a) + \gamma \mathbb{E}_{s'~P(.|s,a)}\ab[\max_{a'\in \mathcal{A}}Q(s',a')]$
      \end{definition}
      Now since $\pi^*(s)=\argmax_a Q^*(s,a)$ implies that thus 
      \[Q^*(C,N) = 1+\gamma \max_{a'}\ab[Q^*(B,a')],\,Q^*(C,S) = 1+\gamma \max_{a'}\ab[Q^*(C,a')]\]
      and 
      \[Q^*(B,N) = -1+\frac\gamma2\ab[\max_{a'}Q^*(A,a')+\max_{a'}Q^*(B,a')]\]
      \[Q^*(B,S) = -1+\frac\gamma2\ab[\max_{a'}Q^*(B,a')+\max_{a'}Q^*(C,a')]\]
      and  
      \[Q^*(C,N) = 1+\frac\gamma2\ab[\max_{a'}Q^*(A,a')+\max_{a'}Q^*(B,a')],\,Q^*(C,S) = 1+\gamma\max_{a'}\ab[Q^*(B,a')]\]

    \end{subproblem}
  \end{problem} 
  \begin{problem}
    \begin{subproblem}[][ref:2a]
      Def of Bellman optimality implies that 
      \[V_h^*(s) = \max_a\ab[r(s,a)+\mathbb{E}_{s'\sim P(.|s,a)}V_{h+1}]\]
      which implies that since 
      \[\hat \pi_h(s) = \argmax_ar(s,a)+\expect_{s'\sim P(s,a)}\ab[V_{h+1}^*(s')]\]
      if $a=\hat \pi_h(s)$ implies that 
      \[V_h^*(s) = r(s,\hat \pi_h(S)) + \expect_{s'\sim P(s,\hat \pi_h(s))}\ab[V^*_{h+1}(s')]\]
      which implies the inequality holds \qed 
    \end{subproblem}
    \begin{subproblem}
      Asssume for $h$, $P(h)=V_{h}^{\hat\pi}(s)\geq V_{h}^*(s)\forall s$. 

      Thus 
      \[V_h^{\hat\pi}(s)\geq V_{h}^*(s) \implies \expect_{s'\sim P(\cdot|s,\hat\pi(s))}\ab[V_h^*(s)]\geq \expect_{s'\sim P(\cdot|s,\hat\pi(s))}\ab[V_h^*(s)]\] 
      which implies that 
      \[r(s,\hat\pi_h(s))+ \expect_{s'\sim P(\cdot|s,\hat\pi(s))}\ab[V_h^*(s)]\geq r(s,\hat\pi_h(s)) + \expect_{s'\sim P(\cdot|s,\hat\pi(s))}\ab[V_h^*(s)] = V_{h-1}^*(s)\forall s\] 
      Now given 2a implies 
      \[V_{h-1}^{\hat\pi}(s) = r(s,\hat\pi(s)) + \expect_{s'\sim P(\cdot|s,\hat\pi(s))}\ab[V_h^*(s')] \]
      And thus $V_{h-1}^{\hat\pi(s)}\geq V_h^*(s)\forall s$ 
       
      And since 2a implies given $h=H$ that $V_h^*(s)\leq r(s,\hat \pi(s)) \forall s$ which implies that by definition since $V_h^{\hat\pi}(s)= r(s,\hat\pi(s))\forall$ implies that $V_h^{\hat\pi}(s)\geq V_h^*(s)\forall s$ given $h=H$. Thus the base case is satisifed ans since we showed the recursive step implies by mathematical induction all cases are satisfied. \qed 
    \end{subproblem}
    \begin{subproblem}
      Assume that $V_h^\pi = V_h^*(s)\forall s$. Then
      \[V_{h-1}^\pi(s) = r(s,\pi_{h-1}(s))+\expect_{s'\sim P(\cdot |s,\pi_{h-1}(s))}\ab[V_h^\pi(s')]\] 
      And $V^\pi$ being bellman optimal implies $\pi$ is as well so 
      \[V_h^*(s)=r(s,\pi_{h-1}(s)) + \expect_{s'\sim P(\cdot|s,\pi_{h-1}(s))}\ab[V_h^*(s')]\forall s \implies V_{h-1}^\pi(s)=V_{h-1}^*(s)\forall s\] 
      
      Now since given $h=H$ implies that $V_h^*(s)=r(s,\pi^*(s)) \forall s$. This implies that since $V^\pi(s)$ is bellman optimal implies that 
      \[V_h^*(s)=r(s,\pi_h(s)) + \expect_{s'\sim P(\cdot|s,\pi_h(s))}\ab[V_{h+1}^*(s')]\forall \] 
      So $V_h^\pi(s) = r(s,\pi_h(s))$ forall $s$. Thus we have shown the base case and the recursive step is true which implies by mathematical induction all cases are satisfied \qed  
    \end{subproblem}
  \end{problem} 
  \begin{problem}
    \begin{subproblem}
      Now because$d_h^{\mu_0}= (P^h_\pi)^T \mu_0$, since $\mu_0$ and $P_{\pi}$ are probability distribtions and transition matricies, it follows that all their entries are nonnegative and sum to $1$. Now since $P_\pi P_\pi$ is a matrix multiplication implies that each entry is the sum of a series of multiplcations of entires in $P_\pi$. Since all entires are nonnegative implies $P_\pi P_\pi$ is nonnegative. This implies that $P_\pi^h = P_\pi P_\pi^{h-1}$ has nonnegative entires. Thus trivially $(P_\pi^h)^T$ has nonnegative entires, which implies since similarly $(P_\pi^h)^T$ and $\mu_0$ have nonnegative entires the product will as well. Thus $d_\gamma^{\mu_0}(s)\geq 0 \forall s$. We also know that since $d_h^{\mu_0}$ is a probability distribtion implies that $\sum_{s\in S\forall s}d_h^{\mu_0} = 1$ 

      Now since 
      \[\sum_{s\in S \forall s}d_\gamma^{\mu_0} = (1-\gamma)\sum_{h=0}^\infty \gamma_h \sum_{s\in S\forall s}(P_\pi^h)^T \mu_0(s) = \]
      \[(1-\gamma)\sum_{h=0}^\infty \gamma h \sum_{s\in S\forall s}d_h^{\mu_0}(s) = (1-\gamma) \sum_{h=1}^\infty \gamma_h = (1-\gamma)\frac{1}{1-\gamma} = 1\]
      Then 
    \end{subproblem}
    \begin{subproblem}
      Let $C = \frac1{1-\gamma}$. 
    \end{subproblem}
    \begin{subproblem}
      Now \[R^\pi(s) = \expect_{a\sim \pi(\cdot | s)}\ab[r(s,a)] = \sum_{a'}\pi(a'|s)r(s,a') = \V{R^\pi(s_1)]\\ \vdots \\ R^\pi(s_n)}\]

      And 
      \[P_\pi(s'|s)=\expect_{a\sim\pi(\cdot|s)}\ab[P(s'|s,a)]=\sum_a \pi(a|s)P(s'|s,a) = \]
      \[\V{
        P_\pi(s_1|s_1) & P_\pi(s_2|s_1) & \cdots & P_\pi(s_n|s_1) \\ 
        P_\pi(s_1|s_2) & P(\pi(s_2|s_2)) & \cdots & P_\pi(s_n|s_2) \\ 
        \vdots & \vdots & \ddots & \vdots \\ 
        P_\pi(s_1|s_n) & P_\pi(s_2|s_n) & \cdots & P_\pi(s_n|s_n)
      }\]

      Which implies that  
      \[V^\pi(s) = \expect_{a\sim\pi(\cdot|s)}\ab[r(s,a)+\gamma\sum_{s'}P(s'|s,a)V^\pi(s')] = \sum_a\pi(a|s)\ab[r(s,a)+\gamma\sum_{s'}P(s'|s,a)V^\pi (s')]\]
      \[= \sum_a \pi(a|s)r(s,a)+\gamma\sum_{s'}\sum_a\pi(a|s)P(s'|s,a)V^\pi(s') = R^\pi(s)\gamma \sum_{s'}P_\pi (s'|s)V^\pi(s') \implies\]
      that $V^\pi = R^\pi + \gamma P_\pi V^\pi$ 
    \end{subproblem}
  \end{problem}
  \begin{problem}
    \begin{subproblem}
      Given 
      \[\ab|Q^{\pi^{h+1}}(s,a)-Q^*(s,a)| =\]
      \[\ab|r(s,a) + \gamma\expect_{s'\sim P(\cdot |s,a)}Q^{\pi^{h+1}}(s',\pi^{t+1}(s')) - r(s,a) - \gamma\expect_{s\sim P(\cdot | s,a)}\max_{a'}Q^*(s',a')| = \]
      \[\gamma\expect_{s'\sim P(\cdot |s,a)}\ab|Q^{\pi^{h+1}}(s',\pi^{t+1}(s'))-\max_{a'}Q^*(s',a')|\]
      \[ = \gamma\expect_{s'\sim P(\cdot |s,a)}\ab|Q^{\pi^{h+1}}(s',\pi^{t+1}(s'))-Q^*(s',\pi^*(s'))|= \]
      \[ = \gamma\expect_{s'\sim P(\cdot |s,a)}\ab|Q^{\pi^{h+1}}(s',\pi^{t+1}(s')) -Q^*(s',\pi^{t+1}(s'))+Q^*(s',\pi^{t+1}(s'))-Q^*(s',\pi^*(s'))|\implies \]
      \[ = \gamma\expect_{s'\sim P(\cdot |s,a)}\ab|Q^{\pi^{h+1}}(s',\pi^{t+1}(s')) -Q^*(s',\pi^{t+1}(s'))+Q^*(s',\pi^{t+1}(s'))-Q^*(s',\pi^*(s'))|\]
      \[\leq\gamma\expect_{s'\sim P(\cdot |s,a)}\ab|Q^{\pi^{h+1}}(s',\pi^{h+1}(s')) - Q^*(s',\pi^{h+1})|\]
      since optimal policy implies that $Q^*(s',\pi^*(s'))\geq Q^*(s',\pi^{t+1}(s')) \implies Q^*(s',\pi^{t+1}(s'))- Q^*(s',\pi^*(s')) \leq 0$.
      
      Now since 
      \[Q^{\pi^{h+1}}(s',\pi^{h+1}(s')) - Q^*(s',\pi^{h+1}) = \gamma\expect_{s''\sim P(\cdot |s', \pi^{h+1}(s'))}\ab[Q^{\pi^{h+1}}(s'', \pi^{h+1}(s''))-Q^*(s', \pi^*(s'))]\]
      implies a recursion such that
      \[\ab|Q^{\pi^{h+1}}(s,\pi^{h+1}(s)) - Q^*(s,\pi^{h+1})| \leq \gamma \ab|\expect_{s'\sim P(\cdot|s,a)}Q^{\pi^{h+1}}(s',\pi^{h+1}(s')) - Q^*(s',\pi^{h+1})| \leq \]
        \[\ab|\gamma^2\expect_{s'\sim P(\cdot|s,a)}\expect_{s''\sim P(\cdot|s',\pi^{h+1}(s')} 
Q^{\pi^{h+1}}(s'',\pi^{h+1}(s'')) - Q^*(s'',\pi^{h+1})|\leq \cdots \leq \gamma^\infty (\expect\cdots)= \gamma^\infty C = 0 \]
      since $\gamma$ is bounded by $[0,1)$ so $\lim_{n\to\infty}\gamma^n = 0$ and since $0\leq \ab|Q^{\pi^{h+1}}(s,\pi^{h+1}(s)) - Q^*(s,\pi^{h+1})|$ implies that   
      \[Q^{\pi^{h+1}}(s,\pi^{h+1}(s)) = Q^*(s,\pi^{h+1})\forall s,a\]
    \end{subproblem} 
    \begin{subproblem}
      %Given finite $A,S$ implies cardinality of policies is bounded by $A^S$.
      Monotonic improvment of PI implies that $Q^{\pi^{h+1}}(s,a)\geq Q^{\pi^h}(s,a)\forall s,a$. 

      Now given the inequality, suppose $Q^{\pi^{h+1}}(s,a) = Q^{\pi^h}(s,a)$. Then 4a implies tha t $\pi^{h+1}$ is an optimal policy. 

      Now suppose $Q^{\pi^{h+1}}(s,a) = Q^{\pi^h}(s,a)$. Thus $\forall s$ since there exists $|A|$ actions implies the number of times $s$ can be improved by $\pi^h$ is bounded by $A|$. Now since $A$ actions $S$ states imply a finite bound of $|A|^|S|$ policies given that the number of total interations is finite implies that there exists a finite point where any more iterations contiutally yields $Q^{\pi^{h+1}}(s,a)=Q\pi^h(s,a)$ which implies that $\pi^h$ is optimal \qed 
    \end{subproblem}
  \end{problem}
  \end{document}
