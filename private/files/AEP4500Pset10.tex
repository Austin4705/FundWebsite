\documentclass[12pt]{amsart}
\usepackage{austin}


\title{AEP 4500 Pset 10}
\author{Austin Wu}

\begin{document}

\maketitle
\begin{enumerate}
  \item Question 1:
  \begin{enumerate}
    \item Recall that the density of states per volume for 3D electron gas is
      \[D_\text{3DGas}(E\geq0)= \frac{(2m)^\frac{3}{2}}{2\pi^2\hbar^3}\sqrt{E}\]
      Which represents the conduction band with the exception of that the bottom of the band is $E_c$ and have effective mass $m_e$. This implies DOS is
      \[D_c(E\geq E_c) = \frac{(2m_e)^\frac{3}{2}}{2\pi^2\hbar^3}\sqrt{E-E_c}\]
      Which implies at fixed $\mu$, 
      \[n(T) = \int_{E_c}^\infty dE D_c(E)n_F(\beta(E-\mu)) = \int_{E_c}^\infty dE \frac{D_c(E)}{e^{\beta(E-\mu)}+1}\]
      Now since $\beta(E-\mu) \gg 1$ implies
      \[\frac{1}{e^{\beta(E-\mu)}+1} \approx e^{-\beta(E-\mu)}\implies\]
      \[n(T) \approx \int_{E_c}^\infty dE D_c(E)e^{-\beta(E-\mu)} = \frac{(2m_e)^\frac{3}{2}}{2\pi^2\hbar^3} \int_{E_c}^\infty dE \sqrt{E-E_c}e^{-\beta(E-\mu)}\]
      Evaluating this integral given $x=\sqrt{E-E_c}$ we get
      \[\int_0^\infty dx\sqrt{x}e^{-\beta x} = -\displaystyle\frac{\partial}{\partial b}\sqrt{\pi}{\beta} = \frac{1}{2}\beta^\frac{-3}{2}\sqrt{\pi} \implies\]
      \[n(T) = \frac{1}{4}(\frac{2m_ek_BT}{\pi\hbar^2})^\frac{3}{2}e^{-\beta(E_c-\mu)} = 2(\frac{m_ek_BT}{2\pi\hbar^2})^\frac{3}{2}e^{-\beta(E_c-\mu)}\]
    \item Similarly to the other case 
    \[D_c(E \leq E_v) = \frac{(2m_h)^\frac{3}{2}}{2\pi^2\hbar^3}\sqrt{E_v-E}\]
    Which implies at fixed $\mu$, 
    \[n(T) \int_0^\infty dE D_c(E)n_F(\beta())\]
  \end{enumerate}
\end{enumerate} 


\end{document}
