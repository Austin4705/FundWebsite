\documentclass[12pt]{amsart}
\usepackage{austin}

\title{Math 6302 Pset 4}

\author{Rishi Gujjar}

\begin{document}
  \maketitle
  \begin{problem}
    \begin{subproblem}
      To begin we show addition 
      \[\sigma_k(r_1+r_2) = (r_1+r_2)(e^{(2k-1)\pi i/n} = r_1e^{(2k-1)\pi i/n} + r_2 e^{(2k-1)\pi i/n} = \sigma_k(r_1)+\sigma_k(r_2)\]
      And also that multiplication holds
      \[\sigma_k(r_1r_2) = (r_1r_2)(e^{(2k-1)\pi i/n}) = (r_1)(e^{(2k-1)\pi i/n})\cdot (r_2)(e^{(2k-1)\pi i/n}) = \sigma_k(r_1)\cdot \sigma_k(r_2)\]
      Note that this works since $e^{(2k-1)\pi i/n}$ is a root of $x^n+1$ and since any multiple of $x^n+1$ evaluates to 0 we multiplicativity.

      Now to show bijectivity, 

      Suppose $\exists r\in \Z[x]/(x^n+1)\st \sigma_k(r)=0$ with $r(x)=r_0 + r_1 x +\cdots + r_{n-1}x^{n-1}$. Then 
      $\sigma_k(r) = r (e^{\pi i /n}) = 0$ but since if $p(e^{\pi i/n}) = 0$ for a integer polynomial $p$, then $p$ is divisible by $x^n+1$. Thus in $\Z[x]/(x^n+1)$, $r(x)\equiv 0\mod x^n+1\implies r=0$. Thus since the kernel of $\sigma_k$ is just 0 which implies injectivity.

      Now also since any element in $\Z[e^{\pi i/n}]$ can be written as 
      \[r_0 + r_1 e^{\pi i/n} + \cdots + r_m(e^{\pi i/n})^m\]
      With $r_i\in \Z, m\in \N$. Thus since $e^{\pi i n/n } = -1$ implies that every element in $\Z[e^{\pi i /n}]$ can be expressed as a linear combination of integer coefficents of the basis 
      \[1, e^{\pi i/n}, \cdots, (e^{\pi i/n})^{n-1}\]
      Now let $r(x)=r_0 + r_1x +\cdots + r_{n-1}x^{n-1}\in \Z[x]/(x^n+1)$ represented by 
      $\sigma_k(x)=(e^{(2k-1)\pi i/n})^m$ such that $(2k-1)m\equiv 1\mod 2n$.
      Thus \[\sigma_k(r(x^m)) =r_0 + r_1 e^{\pi i /n} + \cdots + r_{n-1} (e^{\pi i/n})^{n-1}\]
      %which is the same representation as the elements in $\Z[e^{\pi i/n}]$. 
      Thus any element in $\Z[e^{\pi i/n}]$ is in the image of an $r\in \Z[x]/(x^n+1)$ which implies surjectivity.

      Thus since we have shown bijectviity,  additivity, and multiplicativity which implies $\sigma_k$ is a ring isomorhism. \qed
    \end{subproblem}
    \begin{subproblem}
      For any ring homomorphism that maps to $\C$ 
      \[\sigma^*(0)=\sigma^*(x)^n+\sigma^*(1)\]
      Thus this implies that $\sigma^*(x)^n + 1 = 0$ since $\sigma^*(0)=0, \sigma^*(1)=1$ by definition of a ring homomorphism. 

      This implies that $\sigma^*$ is a root in $x^n+1=0$. Thus since there is a bound of $n$ roots implies that there exists some $k\in [n]$ such that 
      \[\sigma^*(x)=e^{(2k-1)\pi i/n}\]
      This implies that for any element $r\in \Z[x]/(x^n+1)$ that 
      \[\sigma^*(r)=\sum_{i=0}^{n-1}r_i(\sigma^*(x))^i\]
      and 
      \[\sigma_k(r)=\sum_{i=0}^{n-1}r_i(\sigma_k(x))^i\]
      Finally since $\sigma_k(r)=\sigma^*(r)$ then they are equivalent \qed


      %\[\sigma^*(0)=\sigma^*(x)^n + \sigma(1)\implies 
      %Since $\sigma_k: \Z[x]/(x^n+1)\to S$ for $S\subseteq \C$ is a ring homomorphism. Then there exists $k\in\{1,\cdots, n\}$ such that $\sigma^*(x)=\sigma_k(x)$
    \end{subproblem}
    \begin{subproblem}
      Given $r\in \Z[x]/(x^n+1)$, then
      \[||\sigma(r)||^2 = \sum_{k=1}^n |\sigma_k(r)|^2=\sum_{k=1}^n \sigma_k(r)\overline{\sigma_k(r)}\]
      Now since $\sigma_k(r) = \sum_{i=0}^{n-1}r \omega_k^i$ and $\overline{\sigma_k(r)} = \sum_{j=0}^{n-1}r_j w_k^{-j}$ Then 
      \[||\sigma(r)||^2 = \sum_{k=1}^n (\sum_{i=0}^{n-1}r_i \omega_k^i)(\sum_{j=0}^{n-1}r_j\omega_k^{-j}) = \sum_{i,j=0}^{n-1}r_ir_j \sum_{k=1}^n \omega_k^{i-j}\]
    But since $\omega_k = e^{(2k-1)\pi i/n}$ implies that there are $2n$ roots of unity where $\omega_k^n=-1$. This implies that 
    $\sum_{k=1}^n \omega_k^m$ is equal to $n$ if $m\equiv 0\mod n$ and 0 otherwise. 
    % TODO Prove this
    Thus we can simplity our equation to be 
    \[\sum_{i,j=0}^{n-1}r_ir_j \sum_{k=1}^n \omega_k^{i-j} = n\sum_{i=0}^{n-1}r_i^2 = n||r||^2\]
    and thus square rooting each side implies that 
    \[||\sigma_k(r)|| = \sqrt{n}||r||\]\qed
    \end{subproblem}
    \begin{subproblem}
      Let $S=\sigma(r)$ for each $r\in \Z[x]/(x^n+1)) \subset \C^n$. Then for any $\sigma(r_1), \sigma(r_2)\in S$,
      \[\sigma(r_1)+\sigma(r_2)=(\sigma_1(r_1)+\sigma_2(r_2), \cdots, \sigma_n(r_1)+ \sigma_n(r_2))\]
    And since $\sigma_k$ is a ring homomorphism implies 
  \[=\sigma(r_1+r_2)\]
  
  Similarly, 
      \[\sigma(r_1)\sigma(r_2)=(\sigma_1(r_1)\sigma_2(r_2), \cdots, \sigma_n(r_1) \sigma_n(r_2))\]
    And since $\sigma_k$ is a ring homomorphism implies 
  \[=\sigma(r_1r_2)\]

  Now suppose there is a $r_1 r_2 = 0$. Then since $\C$ has no zero divisors, $\forall k$, $\sigma_k(r_1)=0$ or $\sigma_k(r_2)=0$. But due to the fact that $\sigma_k$ is injective  in 1.1 so then $\sigma_k(r)=0\implies r=0$. Thus $S$ has no nontrivial zero divisors.

  Define a map between the two as $z\in \sqrt{n}\Z^n$ as 
  \[A:z\mapsto (\sigma_1(z\frac{1}{\sqrt{n}}), \cdots, \sigma_n(z\frac{1}{\sqrt{n}}))\]
  Then this map is linear since we showed that $\sigma_k$ is linear by definition. We also know this map is length preserving since 
  \[(\sigma_1(z\frac{1}{\sqrt{n}}), \cdots, \sigma_n(z\frac{1}{\sqrt{n}})) = ||\sigma(z\frac{1}{\sqrt{n}})|| = \sqrt{n} ||z\frac{1}{\sqrt{n}}||= ||z||\]


    \end{subproblem}
  \end{problem}
  \begin{problem}
    \begin{subproblem}
      We claim that the explicit generator for $s$ that makes  $\mathcal{I}$ ideal is $s=1$.
      Now let $r_1=x, r_2=-3x-2$. Then 
      \[5r_1+r_2(1+x) 5x + (-3x-2)(1+x) = 5x -3x-3x^2-2-2x = -2-3x^2\]
      But since we are working in $\Z[x]/(x^2+1)$ implies that $x^n=-1, n=2$ so $-2-3x^2 = 3-2 =1 = s$. Thus $s\in \mathcal{I}$.
      Now let $r_1=1=s, r_2 = 0$. Thus $5\cdot1 + 0(1+x) = 5$ Now let $r_1=0,r_2=1=s$. Thus $5\cdot0+1(1+x) = 1+x$. Thus we have shown $x+1$ and $5$ are contianed in the ieal generated by $s$. Thus we have shown both properties and $\mathcal{I}$ is a principal ideal with explicit generator $s=1$.\qed
    \end{subproblem}
    \begin{subproblem}
      Assume that it is a principle ideal. Then $\mathcal{I} = \{2r_1 + r_2(1+x)\}, r_1,r_2\in R$. Then if it is principle there exists $s\in R$ such that $s = a+bx$, $a,b\in \Z$. So then if $I=(s)$, $s$ divides every element inthe ideal. This implies that $s$ divides $2, 1+x$. 

      Now using multiplictive norms, let 
      \[N:R\to \Z, N(a+bx)=a^2 3b^2\]

      Then the norms of the divisibility relations are  
      \[N(s)|N(2)=4, N(s)|N(1+x)=4\]
      So $N(s)\in \{1,2,4\}$.

      Now since $N(s)=2$ is impossible since $a^2+3b^2\equiv 0,1 p\mod 3$ 

      Now since $N(s)=4$ gives $b=0, |a|=2$ so $s\pm 2$ but if $s\pm 2$ then $\pm2|(1+x)$ implies $(1+x)/2 \in R$ which is not true. And if $|a|=|b|=1$ then $s=\pm (1\pm x)$ and $2=sq$, $q$ being the quotient so $N(s)N(q) = 4 = 4N(q)$ so $N(q)=1,q=\pm 1$. Thus $N(s)\neq 4$.

      This means that $N(s)$ must be $1$ which implies that $s=\pm 1$. This would imply that any common divisor of 2 and $1+x$ is a unit. 

      Now suppose $1\in \mathcal{I}$. 
      Then $\exists r_1,r_2\in R \st 1=2r_1+r_2(1+x)$. But since every element of $R=\Z[x]/(x^2+3)$ can be uniquely wirtten 
      as $a+bx$, $a,b\in \Z$ then let $r_1=a+bx, r_2 = c+dx$ implies that 
      \[1 = 2a+2bx+c+dx+cx+dx^2\]
      So then since $x^2=-3$ in $R$, $dx^2=-3d$. Thus 
      \[1 = (2a+c-2d) + x(2b+c+d)\]
      This gives a system of two equations  
      \[1=2a+c-3d\]
      \[2b+c+d=0\]
      And since they have the same partiy both of these equations cannot satisfy, hence a contradiction. and $1\notin \mathcal{I}$.

      Now since we showed that if $I=(s)$ was principal that its generator is a common divisor of $2$ and $1+x$ so $s=\pm1$ and $\mathcal{I}=R$ but $\mathcal{I}!=R$ so we have a contradiction. Thus it is not a principle ideal.\qed
    %Thus for some $h,k\in R, 2 = gh, 1+x = gk$. 
    %Now let $s=a+bx$ and $h=c+dx$. Then $2=gh$ implies that 
    %$2 = (ac-3bd)+(ad+bc)x$. Thus $ad+bc = 0, ac-3bd = 2$. But for this to be an element of $R$, division of $2/(ac-3bd)$ must be an integer so $c=2a/(a^2+3b^2), d=-2b/(a^2+3b^2)$. So $N=a^2+3b^2$ divides $2a, 2b$. Thus
    %If $b=0$ then $N=a^2$ divides $2a$ so $a=\pm1,\pm2$ but $a\pm1\implies s=\pm1$ is not possible since $s$ is not a unit since $\mathcal{I}\neq R$. Thus $a=\pm2\implies s=\pm2$.
    %Now if $a=0$ then $N=3b^2$ and $s=bx$ which implies that $|3b|=1, |3b=2|$ but since they have no integer solutions this is not possible.
      %Now if $a\neq0,b\neq 0$ then $N= a^2+3b^2$ then 
      %Assume that it is ideal. Then there exists $s,a,b\in R$ such that $2=s\cdot a, 1+x = s\cdot b$. This directly implies that $s$ is a common divisor of $2,1+x$.
    \end{subproblem}
    \begin{subproblem}
      Now since if $n$ is a power of two the polynomial $x^n+1$ becomes irreducible implies that there are no factors over $\Z$. This then implies that the quotient ring $R = \Z[x]/(x^n+1)$ does not contian a zero divisor. 

      Now since $s_1R = s_2R$ implies that $s_1$ divides $s_2 $ $s_2 = us_1$ and that $s_2$ divides $s_1$ $s_1 = vs_2$. This implies that $s_1 = v(us_1) = (vu)s_1$ by associativity. Thus $s_1(1-vu)=0$. But since since the integral domain does not contain a zero divisor and $s_1\neq 0$ then $1-vu=0\implies vu =1$ which implies that $v=u^{-1}$ and $u^{-1}u=1$.\qed
    \end{subproblem}
  \end{problem}
  \begin{problem}
    \begin{subproblem}
      Recall that a hexagonal lattice can be represented with a basis of 
      \[\V{1&-\frac12\\0&\sqrt3 /2 }\]
      Then $||b_1|| = 1 = \lambda_1(\lat)$ and so $\Pi_2(b_2) = \V{0\\\frac{\sqrt3}{2}}$ which is the shortest vector in $\Pi_2(\lat)$. Thus it is a HKZ reduced basis. So doing ghram schmidt on it we trivially get that 
      $||\tilde{b}_1|| = \ab||\V{1\\0}|| = 1$ and then 
      \[||\tilde{b_2}|| = ||\Pi_{b_1^\perp}(b_2)|| = \ab||\V{0\\ \frac{\sqrt3}{2}}|| = \frac{\sqrt3}{2}\]
      So then 
      \[\frac{||\tilde{b}_1||}{||\tilde{b}_2||} = \frac1{\sqrt3/2} = \frac2{\sqrt3}\]
    \end{subproblem}
    \begin{subproblem}
      Recall that minkowskis theorem states 
      \[\lambda_1(\lat) \leq \sqrt{\gamma_j}\vol(L)^{\frac1j}\]
      Now given $j$, consider the $j$ dimensional sublattice $\lat' = \Pi_{n-k+1}(\lat)$. Thus 
      \[\vol(L')=\Pi_{k=1}^j \ab||\tilde{b}_{n-j+k}||\]
      and $\lambda_1(\lat') = ||\tilde{b}_{n-j+1}||$ by the definition of a HKZ reduced basis. We also know that $\gamma_j\leq j\forall j$. This implies that 
      \[||\tilde{b}_{n-j+1}||\leq \sqrt{j}(\Pi_{k=1}^j||\tilde{b}_{n-j+k}||)^{\frac1j}\]
      So raising everying by $j$, diving by $||\tilde{b}_{n-j+1}||$, and rearranging indices implies that 
      \[||\tilde{b}_{n-j+1}||^{j-1}\leq j^{j/2} \Pi_{k=1}^{j-1} ||\tilde{b}_{n-k+1}||\]
      Thus this implies that 
      \[||\tilde{b}_{n-j+1}||\leq j^{j/(2(j-1))} \Pi_{k=1}^{j-1} ||\tilde{b}_{n-k+1}||^{1/(j-1)}\]
      And so squaring everyhing implies that 
      \[||\tilde{b}_{n-j+1}||^2\leq j^{j/(j-1)} \Pi_{k=1}^{j-1} ||\tilde{b}_{n-k+1}^{2/(j-1)}\]\qed
    \end{subproblem}
    \begin{subproblem}
      To begin, lets prove the base case of $j=2$.
      \[||\tilde{b}_{n-2+1}||^2 \leq 2^{2/1} \Pi_{k=1}^{1} \]
    \[||\tilde{b}_{n-k+1}||^2\implies ||\tilde{b}_{n-1}||^2 \leq 4 ||\tilde{b}_{n}||^2\]
      And since Question 3.1 implies that $\frac{||\tilde{b}_n||}{||\tilde{b}_{n-1}||} = \sqrt3/2 \leq 2 \leq \sqrt4$ implies that this is true. Thus the $j=2$ base case holds. Now to induct over $j$. Suppose this holds at $j$. Then to induct over $j$ assume
      \[||\tilde{b}_{n-j+1}||^2 \leq j (\Pi_{k=2}^j k^{1/(k-1)}) ||\tilde{b}_n||^2\]
      is true. Then by equation 1
      \[||\tilde{b}_{n-j}|| \leq (j+1)^{(j+1) /j} \Pi_{k=1}^j ||\tilde{b}_{n-k+1}||^{2/(j)}\]
      \[= (j+1)^{1+1/j} ( \|\tilde{b}_{n-j+1}\|^{2/j} \prod_{i=1}^{j-1} \|\tilde{b}_{n-i+1}\|^{2/j} ) \]
      So by applying the inductive hypothesis to each term we get
      \[\leq (j+1)^{1+1/j} ( \left( j \prod_{k=2}^j k^{1/(k-1)} \|\tilde{b}_n\|^2 \right)^{2/j} \prod_{i=1}^{j-1} \left( (j-i+1) \prod_{k=2}^{j-i+1} k^{1/(k-1)} \|\tilde{b}_n\|^2 \right)^{2/j} )\]
      %TODO: Expand here
      Which implies that by simplyfing evaluating the nested $\Pi_{k=2}^r k^{1/(k-1)}$ terms and applying an exponent of $2/j$ with rearranging we get
      \[= (j+1) \left( \prod_{k=2}^{j+1} k^{1/(k-1)} \right) \|\tilde{b}_n\|^2\]
      Thus by mathematical induction this holds \qed
    \end{subproblem}
    \begin{subproblem}
      Recall that since 
      \[||\tilde{b}_{n-j+1}||^2\leq j \cdot (\Pi_{k=2}^j k^{1/(k-1)})\cdot ||\tilde{b}_n||^2 \implies\]
      \[\frac{||\tilde{b}_{n-j+1}||^2}{||\tilde{b}_n||^2}\leq j \cdot (\Pi_{k=2}^j k^{1/(k-1)})\]
      Now since $2\leq k\leq j\leq n$ since $k$ is always greater than 1 and its exponent is greater than 0 we know that $k^{1/(k-1)} \leq n^{1/(k-1)}\forall 2\leq k\leq n$. Thus 
      \[j \cdot (\Pi_{k=2}^j k^{1/(k-1)}) \leq j\cdot(\Pi_{k=2}^j n^{1/(k-1)}) \leq n \cdot (\Pi_{k=2}^j n^{1/(k-1)}) = n\cdot (n^{\sum_{k=2}^j 1/(k-1)})\]
    And using the fact about harmonic numbers $H_n \coloneq \sum_{i=1}^n 1/i \implies H_n \leq \log(n)+1$ imples that 
    \[n\cdot (n^{\sum_{k=2}^j 1/(k-1)})\leq n\cdot(n^{\log(j)+1})\]
    And since we know that $\log(j)\leq \log(n) \forall 2\leq j \leq n,$
    \[ n\cdot(n^{\log(j)+1}) \leq n\cdot(n^{\log(n)+1}) = n^{\log(n)+2}\]
    Thus we have shown that 
  \[\frac{||\tilde{b}_{n-j+1}||^2}{||\tilde{b_n}||^2}\leq n^{\log(n)+2}\]
    Now since we have shown it for any possible $j$ implies that 
    \[\max_{1\leq j\leq n}\frac{||\tilde{b}_{n-j+1}||^2}{||\tilde{b_n}||^2}\leq n^{\log(n)+2}\]
    \end{subproblem}
    \begin{subproblem}
      Given $(b_1,\cdots,b_i)4$, $||b_1||$ is still equal to $\lambda_1(\lat)$. Similarly, for any $k \in \{1,\cdots i\}, (\Pi_k(b_k), \cdots, \Pi_k(b_i))$ is still an HKZ reduced basis since $(\Pi_k(b_k)),\cdots, \Pi_k(b_n))$ is a HKZ reduced basis.

      Thus both axioms of a HKZ reduced basis still satisfy.

      Now given 
      \[\omega(B) = \max_{j\geq i}\frac{||\tilde{b}_i||}{||\tilde{b}_j||}\]
      Implies that since 
    \[\max_{1\leq j\leq n}\frac{||\tilde{b}_{n-j+1}||^2}{||\tilde{b_n}||^2}\leq n^{\log(n)+2}\]
    That $\omega(B) \leq \sqrt{n^{\log(n)+2}} = n^{\log(n)/2 + 1}$ \qed

%      \[\|\tilde{b}_{n-j+1}\|^2 \leq j \left( \prod_{k=2}^j k^{1/(k-1)} \right) \|\tilde{b}_n\|^2\]
%      Setting $j=n$ implies that 
%      \[||\tilde{b}_1||^2 \leq n (\Pi_{k=2}^n k^{1/(k-1)}) ||\tilde{b}_n||^2\]
%      And since $\Pi_{k=2}^n k^{1/(k-1)} = e^{}
%

      
    \end{subproblem}
  \end{problem}
  \begin{problem}
    A like 20 hours 
  \end{problem}


\end{document}

