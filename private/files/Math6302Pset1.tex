\documentclass[12pt]{amsart}
\usepackage{austin}

\newcommand*{\fakebreak}{\par\vspace{\textheight minus \textheight}\pagebreak}
\title{Math 6302 Pset 1}

\begin{document}
  \maketitle
  \begin{problem}
    \begin{subproblem}[][p1b]
      Given $\sum a_i z_i = 0\mod p \implies$ that given $z_{1\cdots n-1}$ that $z_n = - \sum_{k=1}^{n-1}a_iz_i + kq$, $k$ is an arbitary integer. so $\ab<a,z> = \sum a_iz_i - \sum a_iz_i + kq = 0\mod q$  

      Let $b_i$ be defined as 
      \[b_i \coloneq \begin{cases}
        \V{0\\\vdots\\q} & i=n \\
        \hat e_i + \V{0\\\vdots\\-a_i} & i\neq n
      \end{cases}\]
      Then any vector is written as a linear combination
      \[\hat z = \sum_{i=1}^n c_ib_i = \sum_{i=1}^{n-1} \hat e_i + \V{0\\\vdots\\-a_i} + q = \V{c_1\\\vdots\\c_{n-1}\\ c_nq -\sum_{k=1}^{n-1}a_ic_i}\] 
      which is shown to satisfy the property. \qed
    \end{subproblem} 
    \begin{subproblem}
      Suppose $\hat a = 0$. Then $\forall z\in Z^n$, $\sum z_i\cdot 0 = 0 = 0\mod q \implies$ that all vectors with integer coefficents are in this set so the lattice can be represented with the identity basis $\{\hat e_1 \cdots \hat e_n\}$. 

      Suppose $\hat a \neq 0$. Now suppose some $a_i \neq 0$. Since $a_i\neq 0$ implies an inverse inside $a^{-1}_i\in \Z_q$. Then $a' = a_i^{-1}\hat a$. This implies for any $\ab<\hat z, \hat a'>= a_i^{-1}<\hat z, \hat a> = 0\mod p$. This implies that $\lat_a = \lat_{a'}$ and also that we have constructed a new $a$ with equivalent lattice but $a_i$ is 1. Thus we have constructed a new $\hat a$ with equivalent lattice such that $a_i\neq 0$.  
      
      Now following a similar algoirhtm to $\ref{p1a}$, assume $\hat a$ has some value $a_k=1$. Then let 
      \[b_i \coloneq \begin{cases}
        q\hat e_k & i=k \\
        \hat e_i + -a_i \hat e_k & i\neq k
      \end{cases}\] 
      Then any vector is written as a linear combination
      \[\hat z = \sum_{i\neq k} a_i\hat e_i + (c_kq-\sum_{i\neq k}a_ic_i)\hat e_k\] 
      So then $\ab<\hat a, \hat z> = kq = 0\mod q$

      Thus for any $\hat a \neq 0$ we have shown a reduction to an equivalent $\hat a$ such that some $a_i = 1$. We can then use the algoritm described to create a basis. We also have shown the trivial case $\hat a=0$. Thus all cases are satisifed \qed 
    \end{subproblem}
    \begin{subproblem}
      If the probability of $a$ is chosen at random in $\Z_q^n$ then there are $q^n$ possible choices of $a$. For each of the $q^{n-1}$ choices there is 1 value that makes $<a,z>=0$ so 

      % \[Pr_{a~\Z^n_q}[z\in \lat_a] = \frac{q^{n-1}}{q^n} = \frac{1}{q}\qed\]
    \end{subproblem}
  \begin{subproblem}
    
  \end{subproblem}
  \end{problem}


  \begin{problem}
    \begin{subproblem} 
      Let $y,y'\in \lat$. Then $S_y, S_{y'}$ instersect when there are common points $k+y=x=k'+y'$, where $k,k'\in K$. This implies that the points are all that satisfy $k-k'=y'-y$. Now let $y'' = y'-y$ implies that the points are all that satisfy $k-k'= y''$ for $y''\in \lat$. 

    Now given $K$ symmetric implies if $\exists k\in K$ then $-k\in K$. This implies that given any $k\in K$, for any $-k'$ there is $-k'=k$ so the set of values $k-k' = 2K$. Now since there is a equivalent set of $2K$ points and the same lattice $\lat$ that we want to find when they equal in order to find the cardinality of $|\{y'\in \lat\,:\,S_y\cap S_{y'}\neq \emptyset|$ is equivalent to $|\lat \cap 2K|$\qed
    \end{subproblem}
    \begin{subproblem}
      \[\max(S_y) = \max(K + y) \leq \max(K) + \max(y) \leq  r + max(K) = r+d\] 

      Therefore since all points in $S_r$ have norm at most $r+d$ and by definition the ball contains all points with norm less than $r+d$ all points in $S_r$ must be in the ball. This implies that $S_r$ is a subset of $B^n(r+d)$ so $\vol(B^n(r+d)) \geq \vol(S_r \cup B^n(r+d)) + \vol(S_r\cap B^n(r+d)) = \vol(S_r) + \vol(S_r\cap B^n(r+d)) \geq \vol(S_r)$ since volume is always nonnegative. \qed 
    \end{subproblem}
    \begin{subproblem}
      Now since $ = \vol(\cup_{y\in \lat, ||y||\leq r S_i})$ implies that $\vol(S_r) \geq \sum_{y\in\lat, ||y||\leq r}\vol(S_i)/(k-1) = \vol(S_r) \geq \vol(K) \frac{1}{(k-1)}\sum_{y\in\lat, ||y||\leq r}1$. Now since $|\lat \cap 2K|$ is equal to the number of points from intersecting $S_y$ implies that there is at most $k-1 = |\lat \cap 2K|$ other $S_y$ at most that could have intersecting points inside it. Thus since $\sum_{y\in\lat,||y||\leq r}1 = |\lat \cap B^n(r)|$ implies that 
      \[\vol(S_r) \geq |\lat \cap B^n(r)| \frac{\vol(K)}{|\lat \cap 2K|}\qed\]
    \end{subproblem}
    \begin{subproblem}
       Knowing that 
       \[\vol(S_r) \geq |\lat \cap B^n(r)| \frac{\vol(K)}{|\lat \cap 2K|}\qed\]
       implies that 
       \[|\lat \cap 2K| \geq \vol(K)\frac{|\lat \cap B^n(r)|}{\vol(S_r)}\] 
       which implies that if
       \[\frac{|\lat\cap B^n(r)|}{\vol(S_r)} \leq \frac{1}{|\lat|}\equiv |\lat| \leq \frac{\vol(S_r)}{|\lat \cap B^n(r)|}\] 
       Then $|\lat \cap B^n(r)| \geq \frac{\vol(K)}{\vol{\lat}}$. 

       Now since $\vol(K) \leq \frac{\vol(S_r)}{|\lat\cap B^n(r)|}$ implies that since $|\lat| < \vol(K)$ implies that 
       \[|\lat| \leq \vol(K)\leq \frac{S_r}{|\lat\cap B^n(r)|}\]
       thus the inequality holds. \qed
    \end{subproblem}
  \end{problem}

    % Problem 3  
    \begin{problem}
    \begin{subproblem}
      Recall that we know how to generate a basis for a lattice given a vector $\hat x=\V{x_1\\x_2}\in\Z^2_q$ such that $\ab<\hat v,\hat x>=0\mod q \implies\hat v\in\lat $ where the lattice points represent the set of solutions of the equation.  

      Now this implies that given that $q\geq1,\, k,z\in\Z$, $z^2+1=kq$, $a=zb\mod q$. If we find $\hat x\in \Z^2_q$ such that 
      \[a x_1 + b x_2 = 0\mod p\]
      gives the same solution set for $a$ and $b$ as 
      \[a = zb \mod q\] 
      Then we can use the formula given in problem 1 to generate the equivalent lattice basis. 
      
      Now since $a=zb \mod p$ is the same as $a-zb\mod 0$ which is in the proper form and $x_1=1, x_2 = -z$. 

      Now given that $x_1 = 1$ and $x_2 = -z$ implies that our basis is using $x_1$ as the nonzero point 
      \[\V{q&z\\0&1}\]\qed  
    \end{subproblem} 
    \begin{subproblem}
      Given that our lattice is 
      \[\V{q&z\\0&1}\]
      implies our vectors $\V{a\\b}$ satisfy given $x,y\in \Z$, 
      \[(qx+zy)^2 + y^2 = q^2x^2+2zqxy+z^2y^2+y^2 = q^2x^2+2zqxy+y^2(z^2+1)=q^2x^2+2zqxy+y^2kq\]
      And so $q^2x^2+2zqxy+y^2kq = q(qx^2+2zxy+y^2k) \mod p = 0$. \qed
    \end{subproblem}
    \begin{subproblem}
      $\lambda_1(\lat) = ||b_1|| \leq \sqrt{2|\lat|)} = \sqrt{2q}$. \qed 
    \end{subproblem}
    \begin{subproblem}
      Knowing that $a^2+b^2$ by the lattice is represetned by $(qx+zy)^2 + y^2 = q^2x^2+2zqxy+z^2y^2+y^2$ where $x,y$ are varible integers.
     
      Now setting $x=0, y=1$ implies that the equation is equal to $y^2q = q\qed$. 
    \end{subproblem}
  \end{problem}


  \begin{problem}
    \begin{subproblem}
      We know that given a rotation matrix we can generate $RB = \V{\ab||b_1|| & \mu\ab||b_1||\\0&\ab||b_2||}$. Now since any vector in the set can be written as 
    \[v=\V{\ab||b_1|| & \mu\ab||b_1||\\0&\ab||b_2||}\V{m\\n} \implies\] 
    \[\ab||v||^2 = (m\ab||b_1||+n\mu\ab||b_1||)^2 + (n||b_2||))^2 = \] 
    \[= m^2\ab||b_1||^2 + n^2\mu^2\ab||b_1||^2 + n^2||b_2||^2 + 2mn\mu||b_1||^2\] 
    Now since at least $m$ or $n$ is nonzero given $n$ zero implies
    \[v = m^2\ab||b_1||^2 \geq ||b_1||^2\]
    and $m$ zero implies 
    \[v = n^2(1+\mu)\ab||b_1||^2 \geq \ab||b_1||^2\]
    And both nonzero nonnegative or both nonzero and negative imply 
    \[= m^2\ab||b_1||^2 + n^2\mu^2\ab||b_1||^2 + n^2||b_2||^2 + 2mn\mu||b_1||^2 \geq m^2\ab||b_1||^2 \geq \ab||b_1||^2\] 
    And one negative both nonzero implies 
    Which also proves the $n\neq0,m\neq0$ case since 
    \[m^2+n^2\mu^2 - 2mn\mu \geq 0 \implies \]
    \[= m^2\ab||b_1||^2 + n^2\mu^2\ab||b_1||^2 + n^2||b_2||^2 - 2mn\mu||b_1||^2 \geq n^2 ||b_2||^2 \geq n^2 ||b_1||^2 \geq ||b_1||^2\] 

    Thus $\ab||b_1||$ is shorter than any other lattice vector in the set for lattice $RB$. And since $RB$ is a rotation lattice implies $\lambda_1(RB)=\lambda_1(B)$ \qed
    \end{subproblem} 
    \begin{subproblem}
      Since hermites constnat implies that in $\R^{2\times2}$, $\lambda_1(\lat)^2 \leq \gamma_2 \ab|\lat| \implies \ab|\lat| \geq \frac{\sqrt{3}}2$ 
      Now given any basis if we find $B$ such that $B$ is gauss reduced then $\ab|B| = \ab|\pm1\ab|B|| = \ab|\ab|R|\ab|B|| = \ab|RB| = \ab|\V{||b_1||&\mu||b_1||\\0&||b_2||}| = ||b_1||||b_2||$.
      This implies that given \ref{p4a}, $||b_1||=\lambda_1(\lat) = 1$ that $\ab|B| = ||b_2||$. This implies that the smallest $b_2$ can be is $\frac{\sqrt{3}}2$. 
      Now since $\ab|B| \geq \frac{\sqrt{3}}2$ since 
      \[\Delta_p = \frac{\vol(\frac{\lambda_1(\lat)}2 B^2_n)}{2^n}\frac{\lambda_1(\lat)^n}{\ab|\lat|}=\]
      \[\Delta_p = \frac{\vol(\frac{\lambda_1(\lat)}2 B^2_2)}{2^2}\frac{\lambda_1(\lat)^n}{\ab|\lat|}\]
      \[\propto \frac{\lambda_1(\lat)^n}{\ab|\lat|}\]
      And since $\lambda_1$ is bound by $1$ which the hexagonial lattice has and its determinant is $\frac{\sqrt{3}}2$, the smallest possible (and still a value greater than 1) implies that this is the highest sphere packing ratio possible. \qed
    \end{subproblem}
  \end{problem}
  \begin{problem}
    A while 
  \end{problem}
\end{document}
