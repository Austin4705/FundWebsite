\documentclass[12pt]{amsart}
\usepackage{austin}

\title{Math 6302 Pset 3 Test V1}

\author{Rishi Gujjar}
\author{Eric Yachbes}

\begin{document}
  \maketitle
  \begin{problem}
    \begin{subproblem}
      %Recall that a basis for the dual lattice is $B^*\coloneq B(B^TB)^{-1}$. Thus if we are given a new lattice $\lat'=\alpha \lat$ then $B^{*'} = \alpha B (\frac{(B^T)}{\alpha}\alpha B)^{-1} = \alpha (B(B^TB)^{-1}) = \alpha B^*$. Thus since we know that given $\lat, \lat^*$ then $\alpha\lat\implies$ the dual lattice is $\alpha\lat^*$. 

      %Now recall that for any lattice $\lat\subset \R^n, t\in \R^n, s>0, \alpha \geq 1$,
      %\[\rho_{\alpha s}(\lat-t)\leq \alpha^n \rho_s(\lat)\]
      %Then since
      %Implies that given 
      %\[\eta_\epsilon(\alpha\lat)= \inf\{s>0:\rho_{\frac1s}(\alpha\lat^*)\leq 1+\epsilon\}\]
      %And the fact that 
      %\[\rho_{\frac{1}{s}}(\lat)=\sum_{y\in\lat}\rho_{\frac{1}{s}}(y)=\sum_{y\in\lat}e^{-\pi ||y||&2 s^2} = \sum_{y\in\lat}e^{-\pi s^2(\sum_{i=1}^ny_i^2)}\]
      %So 
      %\[\rho_{\frac1s}(\alpha\lat) = \sum_{y\in\lat}e^{-\pi s^2 \sum_{i=1}^n \alpha^2y_i^2}\]
Recall that since 
      \[\eta_\epsilon(\lat)\coloneq \inf\{s>0 : \rho_{\frac1s}(\lat^*) \leq 1+\epsilon\}\]
      Now recall in chapter 5.2 we are given an equivalent definition in a remark: 
      \[\eta_\epsilon \coloneq s\st \rho_{\frac1s}(\lat^*) = 1+\epsilon\]
      This implies we need to find the value such that 
      \[\rho_{\frac1s}(\lat^*)=1+\epsilon=\sum_{y\in\lat^*}e^{-\pi||y||^2 s^2} \]
      Now given a lattice $\alpha\lat$, recall that the dual lattice $(\alpha\lat)^* = \frac1\alpha \lat^*$. Thus
      \[\rho_{\frac1s}(\frac1\alpha\lat^*)=1+\epsilon=\sum_{y\in\frac1\alpha\lat^*}e^{-\pi\alpha^2||y||^2 s^2}= \sum_{y\in\frac1\alpha\lat^*}e^{-\pi||y||^2 (\alpha s)^2} = \alpha s\]
    So $s$ becomes $\alpha s$ so \[\eta_\epsilon (\alpha \lat) = \alpha s = \alpha \eta_\epsilon(\lat)\qed\]
      
    \end{subproblem}
    \begin{subproblem}
      Recall that $\rho_s(y-t)=\rho_s(t)\rho_s(y)e^{2\pi <y,t>/s^2}$. Then, 
      \[\rho_s(\lat-y) =  \rho_s(\lat)\rho_s(t)e^{2\pi /s^2 <y,t>}=\sum_{y\in\lat}\rho_s(y)\rho_s(t)e^{2\pi /s^2 <\lat,t>}\]
      \[= \rho_s(t)\sum_{y\in\lat}\rho_s(y)e^{2\pi <y,t>} \]
      Now due to the symmetry of the lattice sum
      \[= \frac12 \rho_s(y)\ab(\sum_{y\in\lat}\rho_s(y)e^{2\pi /s^2 <y,t>}+\sum_{-y\in\lat}\rho_s(y)e^{2\pi /s^2 <y,t>}) = \rho_s(t)\frac12\sum_{y\in\lat}\rho_s(y)\ab(e^{2\pi <y,t>}+e^{-2\pi <y,t>})\]

      However, we note that 
      \[ \cosh(2\pi <y,t>) = \frac12 (e^{2\pi <y,t>}+e^{-2\pi <y,t>})\]
      but $\cosh(x)\geq 2$, so we have 
      \[\geq \rho_s(t)\sum_{y\in\lat}\rho_s(y)\sqrt{e^{2\pi<y,t>}e^{-2\pi<y,t>}} = \rho_s(t)\sum_{y\in\lat}\rho_s(y) = \rho_s(t)\rho_s(\lat)\]\qed
    \end{subproblem}
    \begin{subproblem}
      Recall that 
      \[\eta_\epsilon \coloneq \inf \{ s>0 :  \rho_{\frac1s}(\lat^*) \leq 1+\epsilon\}\]
      which is equivalent to 
      \[ \forall s \leq \frac{\sqrt{\log(2/\epsilon)}/\pi}{\lambda_1(\lat^*)} \implies \rho_{1/s}(\lat^*) \geq 1+\epsilon,\]
      Now let $y$ be the shortest vector in the lattice. This implies that 
      Since
      \[\rho_\frac1s(\lat^*) \coloneq \sum_{w\in\lat^*} \rho_\frac1s w \geq \rho_\frac1s(0) + \rho_\frac1s(y) + \rho_\frac1s(-y)\geq \rho_\frac1s(0)+\rho_\frac1s(y)+\rho_\frac1s(-y) = 1+2e^{-\pi||y||^2 s^2}\]
      Now since we know that it is the shortest vector we know that  
      \[\geq 1+2e^{-\pi\lambda_1(\lat^*)^2s^2}\]
      We can use that 
      \[ s \leq \frac{\sqrt{\log(2/\epsilon)/\pi}}{\lambda_1(\lat^*)}\]
      and see that 
      \[\rho_{1/s}(\lat^*) \geq 1+2e^{-\pi\lambda_1(\lat^*)^2\left(\frac{\sqrt{\log(2/\epsilon)/\pi}}{\lambda_1(\lat^*)}\right)^2} = 1+\epsilon.\]
      Since the statement was equivalent to 
      \[\eta_{\varepsilon}(\lat)\lambda_1(\lat^*) \geq \sqrt{\log(2/\epsilon)/\pi}.\]
    \end{subproblem}     
    \begin{subproblem}
      Not attempted 
    \end{subproblem}
    \begin{subproblem}
      Recall that 
      \[\rho_s(\lat)\coloneq \sum_{y\in\lat}\rho_s(y)\]
      Thus                                       c
      \[\rho_s(\Z^n) = \sum_{y\in Z^n} e^{-\pi ||x||^2/s^2} = \sum_{y\in \Z^n} e^{-\pi \sum_{i=1}^n}y_i^2 /s^2 = \sum_{y\in\Z^n}\Pi_{i=1}^n e^{-\pi/s^2 y_i^2} = e^{-\pi/s^2}\sum_{y\in \Z^n}\prod^n_{i=1}e^{y_i^2}\]
      Note you can interchange the sum and product since both will have each will have $(x_1,x_2, \dots, x_n)$ takes on every value of $\Z^n$ exactly once. 
      \[= \prod_{i=1}^n \sum_{y\in\Z} e^{-\pi/s^2 y^2} = \prod_{i=1}^n \rho_s(\Z) = \rho_s(\Z)^n\]\qed
    \end{subproblem}
    \begin{subproblem}
      Using the same reason as 1c, this statement is equivalent to showing 
      \[s \leq \sqrt{\log(2n/\epsilon)/\pi} \implies \rho_{1/s}(\lat^*) \geq 1+\epsilon.\]
      There are $2n$ shortest vectors coming from $\pm \vec{e}_i$ for $i = 1, \dots, n.$
      We can use the bounding of 1c and that $(\Z^n)^* = \Z^n$ and see
      \[\rho_{1/s}(\lat^*) = \sum_{\vec{z} \in \Z^n}{\rho_{1/s}(\vec{z})} \geq \rho_{1/s}(0) + \sum_{i=1}^{n}{ \rho_{1/s}(e_i)+ \rho_{1/s}(-e_i)} \]
      \[ \geq 1+2n e^{-\pi\left(\log(2n/\epsilon)/\pi\right)^2} = 1+\epsilon.\]
      This shows that we have the inequality.
    \end{subproblem}
    \begin{subproblem}
      Recall the poisson summation formula, 
      \[\rho_s(\lat) = s^n/det(\lat) \rho_{1/s}(\lat^*)\]
      Then for all $s > 0$
      \[\rho_s(\Z) = s \rho_{\frac1s}(\Z^*) \geq s\rho_{1/s}(0) = s.\]
      We can split the sum of $\rho_s(\Z)$ as 
      \[\rho_s(\Z) = \rho(0) + 2 \sum_{n=1}^\infty \rho_s(n) = 1 + 2\sum_{n=1}^\infty \rho_s(n)\]
      Now since $\rho$ is strictly decreasing between $(0,\infty)$ implies that 
      \[\sum_{n=1}^{\infty}{\rho_{s}(n)} \leq \int_{0}^{\infty}{\rho_s(x) dx}\]
      So we know that 
      \[\rho_s(\Z) \leq 1 + \int^\infty_{-\infty}\rho_s(x)dx = s+1\]
      because the integral of the Gaussian with parameter $s$ is $s.$

      Then since $s \leq\rho_s(\Z) \leq (1+s)$ implies that since from the previous question $\rho_s(\Z^n) = \rho_s(\Z)^n\forall s$ that 
      \[s^n \leq \rho_s(\Z^n) \leq (1+s)^n\]\qed
    \end{subproblem}
  \end{problem}
  \begin{problem}
    \begin{subproblem}
      We are given a matrix $A$ with each entry randomly distributed amongst  $\Z_q$ such that the first $n$ columns of the matrix form an invertible matrix.

      Now extract the first $n$ columns of the matrix. Call this matrix $M$. Let $M^{-1}$ be the inverse of this matrix. Let the ramining rows of the matrix be $A'$. Then $A=[M|A']$ is trivally true. Then this implies that $M^{-1}A = M^{-1}[M|A'] = [I_n|M^{-1}A']$. Note this is the same idea as row reduction. Thus since this form is the same as the matrix form created by row reduction and is invertible, it forms a bijection inside 


      Mapping back to original SIS, this solution works as 
      \[Az \cong M^{-1}MAz \mod q \implies\]
      Now since $M^{-1}Az \cong 0 \mod q \implies$
      \[Az \cong M 0 \mod q \implies\]
      \[Az \cong 0 \mod q\]
      Which solves original SIS and hence showes a reduction between the problems \qed
      
    \end{subproblem}
    \begin{subproblem}
      
Let $(A,b)$ be an LWE instance over $\mathbb{Z}_q$, and write
\[
A = 
\begin{pmatrix}
G \\[4pt]
H
\end{pmatrix},
\quad
b =
\begin{pmatrix}
u \\[4pt]
v
\end{pmatrix},
\]
where $G$ is an $n\times n$ matrix that is invertible mod $q$, and $u$ is $n$-dimensional. We know there exists some $s\in \mathbb{Z}_q^n$ and an error vector $e$ such that
\[
b = A\,s + e,
\]
and we partition $e$ accordingly as
\[
e = 
\begin{pmatrix}
e_1 \\
e_2
\end{pmatrix}.
\]

Define
\[
A' := -\,H\,G^{-1},
\quad
b' := v - H\,G^{-1}\,u.
\]
Because $G$ is invertible over $\mathbb{Z}_q$, $H\,G^{-1}$ behaves as a uniformly random matrix, so $A'$ is suitable for normal-form LWE. We note
\[
v = H\,s + e_2,
\quad
u = G\,s + e_1.
\]
Then
\[
b'
= v - H\,G^{-1}\,u
= (H\,s + e_2)\;-\; H\,G^{-1}(G\,s + e_1)
= e_2 - H\,G^{-1}\,e_1 
= e_2 + A'\,e_1.
\]
Hence, $b'$ is of the form $A'\cdot(\text{something}) + (\text{error})$, which matches normal-form LWE if we treat $e_1$ as the new ``secret'' and $e_2$ as the new ``error.''  

Suppose there is an oracle that solves normal-form LWE on input $(A',b')$ and returns $e_1.$ Then from the upper part of the original instance,
\[
u = G\,s + e_1,
\]
we invert $G$ (which is efficient) and get
\[
s = G^{-1}\bigl(u - e_1\bigr).
\]
Thus, once we have $e_1$, we immediately recover the original secret $s$ in polynomial time.
    \end{subproblem}
  \end{problem}
  \begin{problem}
    \begin{subproblem}
      Recall that  
      \[M_{s,r}\coloneq \int_{||x||\geq r}\rho_s(x)dx\]
      Then if we let 
      $s'=\alpha s$ implies that 
      \[M_{\alpha s, r} = \int_{||x||\geq r}\rho_{\alpha s}(x)dx = \int_{||x||\geq r}e^{-\pi ||x||^2/(\alpha s)^2}\]
      And so if we do 
      \[= \int_{||x||\geq r}e^{-\pi ||x||^2/s^2(1-1+\frac{1}{\alpha^2})}=\int_{||x||\geq r}e^{-\pi ||x||^2/s^2}e^{\pi(1-\frac1{\alpha^2})||x||^2/s^2}\]
      %But since our conditions imply that $\alpha > 1, ||x||\geq r$ implies that $\pi(1-\frac1{\alpha^2})||x||^2/s^2 > \pi ||x||^2/s^2 \geq \pi r /s^2 > 0$ 
      we know that 
      \[e^{-\pi ||x||^2/s^2}e^{\pi(1-\frac1{\alpha^2})r^2/s^2} \geq e^{-\pi ||x||^2/s^2}e^{\pi(1-\frac1{\alpha^2})r^2/s^2}\forall ||x||.\]
      Thus we know that 
      \[\int_{||x||\geq r}e^{-\pi ||x||^2/s^2}e^{\pi(1-\frac1{\alpha^2})||x||^2/s^2}\geq \int_{||x||\geq r}e^{-\pi ||x||^2/s^2}e^{\pi r^2/s^2 (1-\frac1{\alpha^2})} \]
      \[= e^{\pi(1-\frac1{\alpha^2})r^2/s^2}\int_{||x||\geq r}e^{-\pi ||x||^2 /s^2}\]
      And since 
      \[M_{s,r}=\int_{||x||\geq r}e^{-\pi ||x||^2 /s^2}\]
      Is trivially known implies that 
      \[=e^{\pi(1-\frac1{\alpha^2})r^2/s^2}M_{s,r}\]
      Thus we have that 
      \[M_{\alpha s, r}\geq e^{\pi(1-\frac1{\alpha^2}r^2/s^2)}M_{s,r}\]\qed
    \end{subproblem} 
    \begin{subproblem}
      Recall that $M_{\alpha s, r}\leq M_{\alpha s, 0} = (\alpha s)^n$. and if $r>0$ then $M_{\alpha s, r}<M_{\alpha s, 0} = (\alpha s)^n$
    Now since we know from 3.1 that 
    \[M_{\alpha s, r}\geq e^{\pi(1-\frac1{\alpha^2}r^2/s^2)}M_{s,r}\]
    So multiplying each side by $e^{-\pi(1-\frac1{\alpha^2}r^2/s^2)}$ implies that 
    \[M_{s,r}\leq e^{-\pi(1-\frac1{\alpha^2}r^2/s^2)}M_{\alpha s, r}\]
    But since we directly know that leq $M_{\alpha s, r}< M_{\alpha s, 0} = (\alpha s)^n$ 
    Directly implies that 
    \[M_{s,r}\leq e^{-\pi(1-\frac1{\alpha^2}r^2/s^2)}M_{\alpha s, r} < e^{-\pi(1-\frac1{\alpha^2}r^2/s^2)}M_{\alpha s, 0} = e^{-\pi(1-\frac1{\alpha^2}r^2/s^2)} (\alpha s)^n\]
    And thus we showed that 
    \[M_{s,r} < e^{-\pi(1-\frac1{\alpha^2}r^2/s^2)} (\alpha s)^n\]\qed
    \end{subproblem}
    \begin{subproblem}
      Recall that 
      \[\frac1{s^n} \int_{||x||\geq r}\rho_s(x)dx = \frac1{s^n}M_{s,r}\] 

      Now since we directly know that 
      \[M_{s,r} < e^{-\pi(1-\frac1{\alpha^2}r^2/s^2)} (\alpha s)^n\]
      Directly implies that 
      \[\frac1{s^n}\int_{||x||\geq r}\rho_s(x)dx < \frac1{s^n}e^{-\pi(1-\frac1{\alpha^2}r^2/s^2)} (\alpha s)^n\]
      Thus since we know that $r>\sqrt{n/(2\pi)}s$ implies that since this equation holds for any $\alpha>1$ setting $\alpha = \sqrt{\frac{2\pi}n}\frac rs$ trivially knowing that this is greater than 1 implies that  
      \[\frac1{s^n}\int_{||x||\geq r}\rho_s(x)dx < \frac1{s^n}e^{-\pi(1-\frac1er^2/s^2)} (\alpha s)^n = e^{-\pi(1-\frac1er^2/s^2)}\alpha^n = \alpha^ne^{-\pi r^2/s^2}e^{\pi r^2/(\alpha s)^2}\]
      Now replacing $\alpha$ with its value implies that 
      \[= (\sqrt{\frac{2\pi}n}\frac rs)^ne^{-\pi r^2/s^2}e^{\pi r^2/((\sqrt{\frac{2\pi}n}\frac rs) s)^2}\]
      \[= (\frac{2\pi er^2}{ns^2})^{n/2} e^{-\pi r^2/s^2}\]\qed


    \end{subproblem}
  \end{problem}
  \begin{problem}
    Let the algorithm be 
    \begin{lstlisting}[escapechar="]
      Algo(B): #input the basis
        B'= LLL("$\delta= 3/4$", B) #Calculate the LLL reduced basis
        "$B^{*'} = B'((B')^TB')^{-1}$" # Calculate the Dual
        "$b_1$" = B'[1] 
        for i in range(1, n):
          "$y = \algo(B',2^{-\frac n2 + i}\cdot||b_1||)$" 
          if and "$y \in$" Span(B) and not "$y==\{0\}$":
            append y to a list
        return the smallest y in the list
    \end{lstlisting}

    Now to stuty time complexity:

    Now since given that finding $B'$ requires polynomial time complexity it has time complexity of $poly(n,l).$ Now since calculating the dual matrix consists of one transpose ($n^2$) operation, one multiplication ($n^3$) operation, one inversion ($n^3$) operation, and one more multiplication ($n^3$) operation, this step is polynomial still. 
  After we run $\algo$ for $n$ steps, resulting in a time complexity of $T(n,l)$ per step. Thus since checking if it is in the span of a lattice is polynomial (checking by multiplying with the basis of the dual lattice) and checking if its the zero vector is $n$, We get the time complexity inside the loop is $nT(n,l) + n poly(n,l)$ which implies the total time complexity is 
\[nT(n,l)+ n poly(n,l) + poly(n,l) = nT(n,l)+poly(n,l)\]\qed


    Now to show correctness:

    Recall that given $B\in \R^{n\times n}$ is a $\delta=\frac34$ LLL reduced basis for a lattice $\lat$ then $||\tilde b_i||\geq \lambda_1(\lat)/2^{n/2}$. (HW 2.1)

    Also recall that if $B\in \R^{n\times n}$ is a $\delta=\frac34$ LLL basis for $\lat$ then (Theorem 2.12)
    \[||b_1||\leq \frac{\lambda_1(\lat)}{(\delta-\frac14)^{(n-1)/2}}\]
    Now 
    \[||\tilde b_i||\geq \lambda_1(\lat)/2^{n/2}\implies 2^{n/2}||\tilde b_i||\geq \lambda_1(\lat)\implies 2^{n/2}||b_1|| \geq \lambda_1(\lat)\]
    and 
    \[||b_1||\leq \frac{\lambda_1(\lat)}{(\delta-\frac14)^{(n-1)/2}}\implies (\delta-\frac14)^{(n-1)/2}||b_1||\leq \lambda_1(\lat)\implies\]
    \[(\frac12)^{(n-1)/2}||b_1||\leq \lambda_1(\lat)\implies 2^{-(n-1)/2}||b_1||\leq \lambda_1(\lat) \]


    Thus since $||\tilde b_i||\geq \lambda_1(\lat)/2^{n/2}$ implies that there exsits some $i\in Range(1, n)$ such that  $||\tilde b_i||\geq \lambda_1(\lat)/2^{-n/2+i}$ suffices (setting $i=n$ returns the orignal equation). Let $i'$ be the $i$ such that the norm is minimial. 

    Now if $i'=1$ then we know that $2^{-(n-1)/2}||b_1||\leq \lambda_1(\lat)$ and that $\lambda_1(\lat)\leq 2^{-n/2+1}$  which by defintiion implies that 
    \[\lambda_1\leq ||b_1|| \leq 2^{0}||b_1|| \leq \sqrt{2}]\lambda_1(\lat)\]
    And thus $b$ for $i'=1$ is in the range to output the right vector.

    Now if $i'>1$ we know by minimality of $i'$ that 
    \[2^{-n/2+i'}||b_1||\geq \lat_1(\lat)\geq 2^{-n/2 + i'-1}\]
    so we directly know that 
    \[\lambda_1(\lat) \leq 2^{-n/2+i'}||b_i|| \leq 2 \lambda_1(\lat)\]
    and thus $b$ is in the right range such that the output is the right vector for $i'>1$.

    Thus for all $i'\in Range(1,n)$ there exists a output that gives the answer from $\algo$ and thus since $i'$ is guarenteed to exist in that range this algorithm will work.\qed

  \end{problem}

  \begin{problem}
    A lot like 25 hours 
  \end{problem}


\end{document}

