\documentclass[12pt]{amsart}
\usepackage{austin}

\title{Math6302}

\begin{document}
  \maketitle
  \begin{problem}
    \begin{subproblem}
      Given a $\lat$, run the $\gamma-$svp on it. Thus we have obtained a non-zero vector $v$ such that 
      \[||v|| \leq \gamma \lambda_1(\lat)\]
      And since we know from minkowskis theorem that 
      \[\lambda_1(\lat) \leq \sqrt{n}\det(\lat)^{\frac1n}\]
      Implies that
      \[||v|| \leq \gamma\sqrt{n}\det(\lat)^{\frac1n}\]
      Which implies that running $\gamma$-SVP returns a non-zero vector that satisfies $\gamma\sqrt{n}$-MSVP. Thus since $\gamma$-SVP runs in polynomial time there is trivially a reduction to polynomial time. \qed
    \end{subproblem}
    \begin{subproblem}
      Recall from cauchy schwartz that given 
      \[||v||<\frac1{\delta}\det(\lat)^\frac1n\]
      and 
      \[||w||<\delta\det(\lat^*)^\frac1n\]
      That by cauchy schwartz, since 
    \[|\ab<w,v>|\leq ||w||||v||\]
      Implies that 
      \begin{align}
        |\ab<w,v>|\leq ||v||\ab||w||<& \frac1{\delta}\det(\lat)^\frac1n\delta\det(\lat^*)^\frac1n=(\det(\lat)\det(\lat^*))^{\frac1n}\\
      \intertext{Thus since $\det(\lat)\det(\lat^*)=1$ even for non full rank lattices, we know that}
                                   &= 1^\frac1n \\
                                   &= 1
      \end{align}
      But since by definition of a dual lattice we know that $\ab<w,v>\in \Z$ and $|\ab<w,v>|<1$ there is only one possible solution which is tha t $|\ab<w,v>|=0$\qed
    \end{subproblem}
    \begin{subproblem}
      Let $\det(\lat_i)$ be the determinat of the $i$-th lattice. Let $r_i$ be the rank of the $i$-th lattice. Since we start with a lattice at rank $n$ this implies that $r_i=n-i+1$. Thus since $\algo$ solves $\gamma$-MSVP we know by theorem 1.1 that 
      \[||y_i||\leq \delta\det(\lat_i)^\frac1{r_i}\]
      \[||w_i||\leq \delta\det(\lat^*_i)^\frac1{r_i}\]
      We also know for any given lattice intersecting with $w^\perp_i$ reduces the determinant $\det(\lat_{i+1})=\frac{\det(\lat_i)}{||w_i||}$. Now let $v=\lambda_1(\lat)\in \lat$. 
      Recall also that $\det(\lat_i)\,\det(\lat_i^{*}) = 1$ for every $i$ 

      Let \(v\in\lat\) be a non‑zero vector with \(\lVert v\rVert=\lambda_1(\lat)\).
      Now let $t$ be the smallest index such that
      \[t := \min\Bigl\{\,i:\ab||v||\,\le\, \tfrac{1}{\delta}\,\det(\lat_i)^{1/r_i} \Bigr\}.\] 
      Because \(\delta\ge\sqrt{n}\), by Minkowski such an index exists.
      Thus for all \(j<t\) we have
      \(\lVert v\rVert > \det(\lat_j)^{1/r_j}/\delta\).
      Combining this with the bound on \(\lVert w_j\rVert\) and
      Problem 1.2 gives
      \(\langle w_j , v\rangle \neq 0\) so \(v\notin\lat_{j+1}\).
      But since by construction \(v\in\lat_t\), and at \(i=t\) the problem does apply, so since $\alpha$ guarentees a solution to $\delta$-MSVP,
      \[ \langle w_t , v\rangle = 0, \qquad\text{ i.e.\ } v\in\lat_{t+1}. \]
      \[ \lVert y_{t+1}\rVert \;\le\; \delta\,\det(\lat_{t+1})^{1/r_{t+1}} \;=\;  \delta\,\bigl(\det(\lat_t)/\lVert w_t\rVert\bigr)^{1/(r_t-1)} . \]
      Now since \(\lVert w_t\rVert\le\delta\,\det(\lat_t^{*})^{1/r_t}\)
      and \(\det(\lat_t)\det(\lat_t^{*})=1\),
      \[ \det(\lat_{t+1})^{1/r_{t+1}} \;\le\; \bigl(\det(\lat_t)\bigr)^{1/(r_t-1)} =\Bigl(\delta\,\lVert v\rVert\Bigr)^{\frac{r_t}{r_t-1}}\!, \]
      where the equality uses the definition of \(t\).
      Thus,
      \[ \lVert y_{t+1}\rVert \;\le\; \delta\,\Bigl(\delta\,\lVert v\rVert\Bigr) \;=\; \delta^{2}\,\lambda_1(\lat). \]
      Thus the algorithm outputs the shortest vector among
      \(y_1,\dots,y_n\),
      so the final answer \(y\) satisfies
      \[ \lVert y\rVert\le\lVert y_{t+1}\rVert\le\delta^{2}\,\lambda_1(\lat), \]
      i.e.\ it is a valid solution to \(\delta^{2}\)-SVP.
    \end{subproblem}
  \end{problem}
  \begin{problem}
   %  \begin{subproblem}
   %    Let $y_1,\dots,y_n$ be a set of linear independent vectors that span $\R^n$ such that $y_1,\dots y_l$ forms a basis for $\lat'$ and $y_{l+1},\dots,y_n$ is orthogonal with eachother. This is guarenteed to exist since there exist $y_1,\dots y_l$ vectors that form a basis for $\lat'$, implying they span $\R^n$, and given those $l$ vectors are linearly independent there must exist $n-l$ linearly independent vectors that form the nullspace. Running ghram schmit on these $n-l$ vectors implies there must exist an orthogonal set. Since these vectors span $\R^n$, this implies that for all vectors $w\in \R^n$, there exists a uniuqe set of coefficents in $\R$ such that $w=\sum_{i=1}^nc_iy_i$. 
   %    \begin{align*}
   %      (\lat')^\perp &\coloneq \{w\in \R^n: \forall x\in (\lat'), \ab<w,x>=0\} \\ 
   %        &= \{\sum_{i=1}^nc_iy_i : c\in \R^n: \forall v\in \R^l, \ab<\sum_{i=1}^nc_iy_i,\sum_{i=1}^l v_iy_i>=0\} \\
   %                 \intertext{And by linearity we know that}
   %                 &= \{\sum_{i=1}^nc_iy_i: c\in \R^n: \forall v\in \R^l, \sum_{i=1}^n\sum_{j={l+1}}^nc_iv_j \ab<y_i,y_j>=0\} \\
   %                 \intertext{Thus since for $j>l$, we know that $\ab<y_i,y_j>=\delta_{ij}$ we know that}
   %                 &= \{\forall w\in \R^n: \exists c\in \R^{(n-l)}: w=\sum_{i=l+1}^nc_iy_i\} \\
   % \end{align*}
   % so we get that $(\lat')^\perp$ is directly equal to the linear space spanned by $y_{l+1},\dots,y_n$.
   %
   % Similarly recall from HW0 that $\Pi$ is linear. Let $z_1,\dots,z_n$ be a set of basis vectors such that $z_i$ is linearly dependent to $y_i$ for $i\in\{1,\dots,l\}$. This is guarenteed to exist since $\lat'$ is a sublattice. Thus 
   % \[\Pi_{(\lat')^\perp}(w) = \sum_{i=1}^n c_i \Pi_{(\lat')^\perp}(z_i)\]
   % And since we know that $(\lat')^\perp$ is defined by the set of $y_{l+1},\dots,y_n$ vectors and $y_1,\dots y_l$ defines $\lat'$ we know that $\Pi_{(\lat')^\perp}(z_i)=0$ for any $i\in\{1,\dots,l\}$. We also know that since $y_1,\dots,y_l$ spans $\R^l$ then $z_1,\dots,z_l$ also spans $\R^l$ which implies that $z_{l+1},\dots,z_n$ must span the nullspace of $z_1,\dots,z_l$ since $z_1,\dots,z_n$ spans $\R^n$. Thus since $y_{l+1},\dots,y_n$ spans the nullspace we well $\Pi_{(\lat')^\perp}(z_i)=1$ for any $i\in\{l+1,\dots,n\}$. Thus there must exist some equivlaent construction
   % \[= \sum_{i={l+1}}^nc_i y_i\]
   % and we get that the span of $S$ is equal to the linear span of $y_{l+1},\dots,y_n$ and since both spaces are equivlaent to the linear space spanend by $y_{l+1},\dots,y_n$ we know they are equal.\qed
   %  \end{subproblem}
   %  \begin{subproblem}
   %    Let $y_1,\dots,y_n$ be a set of vectors that span $\R^n$ such that $y_1,\dots y_l$ forms a basis for $\lat'$. This is guarenteed to exist since there exist $y_1,\dots y_l$ vectors that form a basis for $\lat'$, implying they span $\R^n$, and given those $l$ vectors todo
   %
   %    Since these vectors span $\R^n$, this implies that for all vectors $w\in \R^n$, there exists a uniuqe set of coefficents in $\R$ such that $w=\sum_{i=1}^nc_iy_i$. 
   %
   %    Similarly, since $\lat'$ spans $\R^n$ and $y_1,\dots,y_l \in \lat'$ are linearly independent, we know that $y_1,\dots,y_l$ span $\lat'$. Thus since $y_1,\dots,y_n$ spans $\R^n$ we know that $y_{l+1},\dots,y_n$ must form a basis for the nullspace of $\lat'$. Thus since
   %
   %    Similarly recall from HW0 that $\Pi$ is linear. Thus 
   %    \[\Pi_{(\lat')^\perp}(w) = \sum_{i=1}^n c_i \Pi_{(\lat')^\perp}(y_i)\]
   %    And since we know that $(\lat')^\perp$ is defined by the set of $y_{l+1},\dots,y_n$ vectors and $y_1,\dots y_n$ is linearly independent with each other meaning $\Pi_{(\lat')^\perp}(y_i)=0$ for any $i=\{1,\dots,l\}$, we know that
   %    \[= \sum_{i={l+1}}^nc_i y_i\]
   %    Thus we get that the span of $S$ is equal to the linear span of $y_{l+1},\dots,y_n$ and since both spaces are equivlaent to the linear space spanend by $y_{l+1},\dots,y_n$ we know they are equal.\qed
   %  \end{subproblem}

    % \begin{subproblem}
    %   Now recall from the previous provlem that $(\lat')^\perp$ is equal to a span of linearly independent set of vectors $z_{l+1},\dots,z_n$. Where $y_1,\dots,y_n$ is a set of vectors that span $\R^n$ and $y_1,\dots,y_l$ is a basis for $\lat'$.
    %
    %   Now given that $T\coloneq \lat^*\cap (\lat')^\perp$ with $S^*\coloneq \{w\in \mathrm{span}(S):\forall y\in S,\ab<y,w>\in \Z\}$, we know that $(\lat')^\perp$ is constructed by the set of $z_{l+1},\dots,z_n$. Thus since for any $y_1,\dots,y_l\in \lat'$, $\ab<z_i,y_j>=0\in \Z$. Now for any $y_{l+1},\dots,y_n$, we know that
    %
    %   Thus implies that any $z_i$ is in $\lat^*$ which implies that $T=(\lat')^\perp$ and thus trivially implies that $\mathrm{span}(T)=(\lat')^\perp$.
    % \end{subproblem}
    \begin{subproblem}
      Let $y_1,\dots ,y_l\in \lat'$ be a basis of $\lat'$ and extend it to
      $y_{l+1},\dots ,y_n\in \lat$ so that $y_1,\dots ,y_n$ is a basis of $\R^n$.
      Set $U:=\mathrm{span}(\lat')$ and $V:=\mathrm{span}(y_{l+1},\dots ,y_n)$.
      Because $U\oplus V=\R^n$, we have $(\lat')^\perp=V^\perp$ and
      $\dim V^\perp=n-l$. Note that $V\cap U=\{0\}$.
      
      For $i>l$ put $z_i:=\Pi_{(\lat')^\perp}(y_i)$. Since the vectors $z_{l+1},\dots ,z_n$ are linearly independent (projection is injective on V since $\cap U=\{0\}$) and lie in $(\lat')^\perp$, so they form a basis of $(\lat')^\perp$.

      Now for each $z_i$ since we know it belongs to $S=\Pi_{(\lat')^\perp}(\lat)$ we know that $(\lat')^\perp\subseteq\mathrm{span}(S)$.
      The opposite inclusion is also obvious, which implies equality.
    \end{subproblem}


    \begin{subproblem}
    Let $y_{1},\dots ,y_{l}\in \lat'$ be a basis of $\lat'$ and extend it to linearly independent $y_{l+1},\dots ,y_{n}\in \lat$. This is guarenteed to exist since $\lat'$ is a sublattice. Now write $B:=\{y_{1}\;\dots\;y_{n}\}$, so $L=B\Z^{n}$ and its dual lattice has basis $B^{-*}:=B^{-T}$. Denote the dual basis vectors by $y_{1}^{*},\dots ,y_{n}^{*}$, so $B^{-*}\coloneq\{y_{1}^{*}\;\dots\;y_{n}^{*}\}$.
    
    Now since by construction $y_{j}^{*}\in \lat^{*}$. This implies that for $1\le i\le l<j\le n$ we have $\ab<y_{i},y_{j}^{*}>=\delta_{ij}=0$, hence $y_{j}^{*}\perp \lat'$, so $y_{j}^{*}\in(\lat')^{\perp}$. Not since independence is inherited from the dual basis. Thus $y_{l+1}^{*},\dots ,y_{n}^{*}\in \lat^{*}\cap(L')^{\perp}=T$ and the vectors are linearly independent.
    
    There are $n-l$ such vectors, matching
    $\dim(\lat')^{\perp}$, so this implies that $\operatorname{span}(T)\supseteq\operatorname{span}\{y_{l+1}^{*},\dots ,y_{n}^{*}\}=(\lat')^{\perp}$.
    
    Thus since we also know that by definition $T\subseteq(\lat')^{\perp}$, hence
    \[ \mathrm{span}(T)=(\lat')^{\perp}. \]
    \end{subproblem}
    

    \begin{subproblem}
      Let $w\in\lat^*\cap (\lat')^\perp$. For any $y\in \lat$ let $s\coloneq\Pi_{(\lat')^\perp}(y)\in S$. Thus since $w$ is in $(\lat')^\perp$ and $s(y)$ is in the proejction space which is self adjoint implies that for any $y\in\lat$, 
      \[\ab<s(y),w>=\ab<\Pi_{(\lat')^\perp}(y),w> = \ab<y, \Pi_{(\lat')^\perp}(w)>=\ab<y,w>\in \Z\]
      Since $w\in \lat^*$. Thus since $w$ $\mathrm{span}(S)=(\lat')^\perp$ implies that $w\in S^*$.

      Now let \(w\in S^{*}\). By definition of \(S^{*}\) we already have \(w\in\mathrm{span}(S)=(\lat')^\perp\).
      To prove \(w\in \lat^{*}\), pick an arbitrary \(y\in \lat\) and set 
      \(y = y_{\parallel}+y_{\perp}\) with \(y_{\parallel}\in\operatorname{span}(\lat')\) and 
      \(y_{\perp}:=\Pi_{(\lat')^\perp}(y)\in S\).  Then
      \[ \langle y,w\rangle = \langle y_{\parallel},w\rangle + \langle y_{\perp},w\rangle. \]
      Because \(w\in (\lat')^\perp\), the first term vanishes (\(w\perp \lat'\)) and the second term is an integer since
      \(y_{\perp}\in S\) and \(w\in S^{*}\).  Hence \(\langle y,w\rangle\in\mathbb Z\) for all \(y\in \lat\), which implies that\ \(w\in \lat^{*}\).
      Thus the two inclusions give \(S^{*}=lat^{*}\cap (\lat')^\perp\), establishing that
      \[ \bigl(\Pi_{(\lat')^{\perp}}\!(\lat)\bigr)^{*} \;=\; \lat^{*}\cap (\lat')^{\perp}. \]
    \end{subproblem}
  \end{problem}
  \begin{problem}
    Not as much as others
  \end{problem}


\end{document}
