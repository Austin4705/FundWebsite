\documentclass[12pt]{amsart}
\usepackage{austin}

\title{Math 6510 Pset 2}

\begin{document}
  \maketitle
  \begin{problem}[Problem 1]
    \begin{proof}[A]
      Let $C_n(CK)$ be the group of n chains on $CK$. This implies that the boundary map $\partial_n: C_n(CK)\to C_{n-1} (CK)$ maps a simplex to its boundary. 
      
      Now lets define $h: C_n(CK)\to C_{n+1}(CK)$. This implies that $\partial_{n+!}(h(a))= a-h(\partial_n(a))$. since the boundary of a cone over $a$ consists of $a$ and the cone. 

      Now suppose $x\in C_n(CK)$. Then $\partial_n(c)=0$ this implies that $\partial_{n+1}(h(c)) = c - h(\partial_n(c)) \implies \partial_{n+1}(h(c))=c$ since $\partial_n(c)=0$. Thus $c$ is a boundary. Now since every n cycle in $CK$ is a bounadrt, all homology groups are equal to $0$ for $n>0$. 
    \end{proof} 
  \end{problem}  
  
  \begin{problem}[Problem 2]
    \begin{itemize}
      \item  Now suppose $X$, $Y$ are homotopy equiv this implies that $f:X\to Y$, $g: Y\to X$ exist such that $g\circ f ~ id_X, f \circ g ~ id_Y$. This implies that given $f: (X,\varnothing)\to (Y,\varnothing), g:(Y, \varnothing)\to (X,\varnothing)$ that we get the compostion is equivalent to the idetities. Therefore the sets are homototpy equivalent as pairs. 

      Suppose they are homotpy equivlaent as pairs. then the maps between them must compose to be equivalent to the identity elements which implies these maps are homotpies which of the underlying spaces $X, Y$, so they are homotpy equivlaen. Thus iff. 

      Now suppose $B$ is nonempty then there is a map $f:(X\varnothing)\to (Y,B), g:(Y,B)\to (X, \varnothing)$ where the compostion is equivalent to an idenity element. But since $f$ must map $\varnothing$ to $B$, this can only happen if $B$ is nonempoty thus $B$ must be nonempty otherwise contradiction. 

    \item Give homotpy equivalence $f$ implies there are maps such that the compostion is equivalent to $id_{X,A}$ and $id_{Y, B}$. Thus since they are also homotypies of their underlying spaces, they are homotopy equivlaent. 

    \item 
  \end{itemize} 

  \end{problem}

  \begin{problem}[Problem  3]
    Now consider the pair $(C_f, Y)$, $Y$ is a subspace of $C_f$. Now this implies a log exacty sequence must exist that looks like 
    \[\cdots \to H_i(Y)\to H_i(C_f)\to H_i(C_f, Y)\to H_{i-1}(Y)\to \cdots \]
    Now since since $C_f$ is constructed by attatching $X\times I$ to $Y$ via $f$. This implies that $H_i(C_f, Y)$ is isomorpihc to $\Sigma X$. Now since $\hat H)_i(\Sigma X))\equiv \hat H_{i-1}(X), H_i(C_f, Y)\equiv \hat H_{i-1}(X)$. implies that by substituting this isomorpihsm into our long exact sequence implies a sequence of the form requested.  
  \end{problem}

  \begin{problem}[Problem 4]
    \begin{enumerate}
      \item Now since we know the sequence is exact implies the image $\Z /p$ in $G$ is isomorphic to $\Z/p$ and the quotient of $G/(\Z/p)$ is isomorpihc to $\Z/q$. Thus $G$ extends $\Z/p$ by $\Z/q$. 

      Suppose that $p=q$. Then since $G$ is abelian implies it must be finitely generated by a abelian group with cardinality $p^2$. Now since we know for primes the only abelian groups of cardinality $p^2$ are $\Z/p \times \Z/p$ and $\Z/p^2$ implies these are the possible groups for $G$. 

      Now suppose that $p\neq q$. Then lagranges thoerm implies if a common subgroup exists its order must divide both $p,$. But since they are primes implies no nontrivial subgroups. This implies since groups are abelian that $G$ must be isomorphic to $\Z/p \times \Z/q$, therefore it is the only possible group.  
      
    \item 
      Since the sequence is short implies injectivity from $A\to B$ which implies that $Hom(\Z^n, A)\to Hom(\Z^n, B)$ is injective. This implies exactnes at $Hom(\Z^n, A)$.  
      
      Now since the map from $\Z^n\to B$ is in the kernel of $Hom(\Z^n, B)\to Hom(\Z^n, C)$ implies when it is comples with $B\to C$ it maps to 0 implying that it maps to $A$. This impies exactness at $Hom(\Z^n, B)$

      Now since $B\to C$ is surjective given defition of SES implies that $Hom(\Z^n, B)\to Hom(\Z^n, C)$ is surjective. This implies exactness at$Hom(\Z^n, C)$

      Now since we have exactnes at all three locations implies the sequence is exact. \qed  

      \item Suppose instead of $\Z^n$ let the group be $\Z/2$. Then 
      \[0\to \Z/2 \to \Z/4 \to \Z/2\to 0\]
      Implies that since $Hom(\Z/2,\Z/2) \equiv Hom(\Z/2,\Z/4) \equiv \Z/2$ then 
      \[0\to Hom(\Z/2, \Z/2) \to Hom(\Z/2, \Z/4) \to Hom(\Z/2, \Z/2) \to 0\]
      becomes 
      \[0\to \Z/2\to \Z/2 \to \Z/2 \to 0\]
      And so this implies that the maps are identity maps which implies the compositon is not zero, so the sequence is not exact. 
    \end{enumerate}
    
  \end{problem}
  
  \begin{problem}[Problem 5]
  \begin{enumerate}
    \item Since $G$ has only 1 object $*$, $F$ must have the property $F(*)=*$. Now since $g\in G(*,*)=G$ implies that $F(g)\in H(*,*)=H$. Therefore $F(i_G)=i_H$. and $F(g\cdot h)=F(g)\cdot F(h)$. This implies $F$ is a group homomorphism. 

      Now since a natural transofmrtion must hold commutatitivty between it and all morphisms between implies that given $F$ is a group homomorphism implies that a natrual transofmration must consist of $n\circ F(g) = F'(g)\circ a \forall g\in G$ whcih implies that all natural trasformations must look like $F'(g)=n F(g)n^{-1}\forall g\in G$. 

    \item Let $F(X)=X$ and $G(X)=X\cup \{X\}$. Then the commutatitivty diagram for $\alpha(x)=x$ is shown since $\forall x\in X$, 
      \[(G(f)\circ \alpha)(x) = G(f)(x)=f(x)\]
      \[(\alpha \circ f)(x)=\alpha (f(x)) = f(x)\]
      But since $G(\varnothing) = \varnothing \union \{*\} = \{*\}$ and $F(\varnothing)=\varnothing$ implies noncommutativity of any map $\beta$. Thus there is no natural transofmration. 

    \item 
      Suppose $\exists H: C\times I\to D \st H|_{C\times 0} = F, H|_{C\times 1}=G$. Then let $\alpha: F(c)\to G(c)$ by $H(id_c, 0)\to 1$ for any object $c\in C$. Then since $H$ is a functor implies composition of morphisms $C\times I$ which implies for any morphism $f: c\o c'$ The commutatitivty diagram mapping $F(c) \to G(c)$ and $F(c')\to G(c')$ with $\alpha$ and $F(c)\to F(c'), G(c)\to G(c')$ commutes. Therefre there exists a natural transofmration $\alpha$. 

      Now suppose there exists $\alpha$. Then Let $H: C\times I \to D$ such that $H(c,0)=F(c), H(c,1)=G(C)$ and $(f,id_0): (c,0)\to (c',0)$ implies $H(f, id_0)=F(f)$ and $(f,id_1): (c,1)\to (c',1)$ implies $H(f, id_1)=G(f)$. Therefore since $\alpha$ is a natural transofmration $H$ preserves idenntiy and composition properties which implies $H$ is a functor. 

    \item TODO 
  \end{enumerate} 
  \end{problem}

  \begin{problem}[Problem 6]
     Recall that a natural transformtion is a transformtion of functors that preserves "naturally" the internal structure of the categories (The transformation is associative and commutes). 

     This can be shown in $\partial_i$ due to the fact that the boundary map has a commutative structure between $H_i(A)\to H_i(B)), H_{i+1}(X,A)\to H_{i+1}(Y,B)$ and $H_{i+1}(X,A)\to H_i(A), H_{i+1}(Y,B) \to H_i(B)$ since the homology groups act as functors $H$ that transform the categories $(X,A)$. This shows a nature of structure preservance which implies naturality 

     Now since $\Sigma X$ of $X$ is $X\times [0,1]/\sim $, $(x,0)\sim (x',0), (x,1)\sim(x',1)\forall x, x'\in X$ implies that any map $f: X\to Y$ incudes a map $\Sigma X\to \Sigma Y$ such that this map commutes with the map induced by $f$. This directly implies a natural transofmration. 
  \end{problem}
  
\end{document}
